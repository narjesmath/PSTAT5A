% Options for packages loaded elsewhere
\PassOptionsToPackage{unicode}{hyperref}
\PassOptionsToPackage{hyphens}{url}
\PassOptionsToPackage{dvipsnames,svgnames,x11names}{xcolor}
%
\documentclass[
  11pt,
]{article}

\usepackage{amsmath,amssymb}
\usepackage{iftex}
\ifPDFTeX
  \usepackage[T1]{fontenc}
  \usepackage[utf8]{inputenc}
  \usepackage{textcomp} % provide euro and other symbols
\else % if luatex or xetex
  \usepackage{unicode-math}
  \defaultfontfeatures{Scale=MatchLowercase}
  \defaultfontfeatures[\rmfamily]{Ligatures=TeX,Scale=1}
\fi
\usepackage{lmodern}
\ifPDFTeX\else  
    % xetex/luatex font selection
\fi
% Use upquote if available, for straight quotes in verbatim environments
\IfFileExists{upquote.sty}{\usepackage{upquote}}{}
\IfFileExists{microtype.sty}{% use microtype if available
  \usepackage[]{microtype}
  \UseMicrotypeSet[protrusion]{basicmath} % disable protrusion for tt fonts
}{}
\makeatletter
\@ifundefined{KOMAClassName}{% if non-KOMA class
  \IfFileExists{parskip.sty}{%
    \usepackage{parskip}
  }{% else
    \setlength{\parindent}{0pt}
    \setlength{\parskip}{6pt plus 2pt minus 1pt}}
}{% if KOMA class
  \KOMAoptions{parskip=half}}
\makeatother
\usepackage{xcolor}
\usepackage[margin=1in]{geometry}
\setlength{\emergencystretch}{3em} % prevent overfull lines
\setcounter{secnumdepth}{5}
% Make \paragraph and \subparagraph free-standing
\makeatletter
\ifx\paragraph\undefined\else
  \let\oldparagraph\paragraph
  \renewcommand{\paragraph}{
    \@ifstar
      \xxxParagraphStar
      \xxxParagraphNoStar
  }
  \newcommand{\xxxParagraphStar}[1]{\oldparagraph*{#1}\mbox{}}
  \newcommand{\xxxParagraphNoStar}[1]{\oldparagraph{#1}\mbox{}}
\fi
\ifx\subparagraph\undefined\else
  \let\oldsubparagraph\subparagraph
  \renewcommand{\subparagraph}{
    \@ifstar
      \xxxSubParagraphStar
      \xxxSubParagraphNoStar
  }
  \newcommand{\xxxSubParagraphStar}[1]{\oldsubparagraph*{#1}\mbox{}}
  \newcommand{\xxxSubParagraphNoStar}[1]{\oldsubparagraph{#1}\mbox{}}
\fi
\makeatother


\providecommand{\tightlist}{%
  \setlength{\itemsep}{0pt}\setlength{\parskip}{0pt}}\usepackage{longtable,booktabs,array}
\usepackage{calc} % for calculating minipage widths
% Correct order of tables after \paragraph or \subparagraph
\usepackage{etoolbox}
\makeatletter
\patchcmd\longtable{\par}{\if@noskipsec\mbox{}\fi\par}{}{}
\makeatother
% Allow footnotes in longtable head/foot
\IfFileExists{footnotehyper.sty}{\usepackage{footnotehyper}}{\usepackage{footnote}}
\makesavenoteenv{longtable}
\usepackage{graphicx}
\makeatletter
\newsavebox\pandoc@box
\newcommand*\pandocbounded[1]{% scales image to fit in text height/width
  \sbox\pandoc@box{#1}%
  \Gscale@div\@tempa{\textheight}{\dimexpr\ht\pandoc@box+\dp\pandoc@box\relax}%
  \Gscale@div\@tempb{\linewidth}{\wd\pandoc@box}%
  \ifdim\@tempb\p@<\@tempa\p@\let\@tempa\@tempb\fi% select the smaller of both
  \ifdim\@tempa\p@<\p@\scalebox{\@tempa}{\usebox\pandoc@box}%
  \else\usebox{\pandoc@box}%
  \fi%
}
% Set default figure placement to htbp
\def\fps@figure{htbp}
\makeatother

\makeatletter
\@ifpackageloaded{tcolorbox}{}{\usepackage[skins,breakable]{tcolorbox}}
\@ifpackageloaded{fontawesome5}{}{\usepackage{fontawesome5}}
\definecolor{quarto-callout-color}{HTML}{909090}
\definecolor{quarto-callout-note-color}{HTML}{0758E5}
\definecolor{quarto-callout-important-color}{HTML}{CC1914}
\definecolor{quarto-callout-warning-color}{HTML}{EB9113}
\definecolor{quarto-callout-tip-color}{HTML}{00A047}
\definecolor{quarto-callout-caution-color}{HTML}{FC5300}
\definecolor{quarto-callout-color-frame}{HTML}{acacac}
\definecolor{quarto-callout-note-color-frame}{HTML}{4582ec}
\definecolor{quarto-callout-important-color-frame}{HTML}{d9534f}
\definecolor{quarto-callout-warning-color-frame}{HTML}{f0ad4e}
\definecolor{quarto-callout-tip-color-frame}{HTML}{02b875}
\definecolor{quarto-callout-caution-color-frame}{HTML}{fd7e14}
\makeatother
\makeatletter
\@ifpackageloaded{caption}{}{\usepackage{caption}}
\AtBeginDocument{%
\ifdefined\contentsname
  \renewcommand*\contentsname{Table of contents}
\else
  \newcommand\contentsname{Table of contents}
\fi
\ifdefined\listfigurename
  \renewcommand*\listfigurename{List of Figures}
\else
  \newcommand\listfigurename{List of Figures}
\fi
\ifdefined\listtablename
  \renewcommand*\listtablename{List of Tables}
\else
  \newcommand\listtablename{List of Tables}
\fi
\ifdefined\figurename
  \renewcommand*\figurename{Figure}
\else
  \newcommand\figurename{Figure}
\fi
\ifdefined\tablename
  \renewcommand*\tablename{Table}
\else
  \newcommand\tablename{Table}
\fi
}
\@ifpackageloaded{float}{}{\usepackage{float}}
\floatstyle{ruled}
\@ifundefined{c@chapter}{\newfloat{codelisting}{h}{lop}}{\newfloat{codelisting}{h}{lop}[chapter]}
\floatname{codelisting}{Listing}
\newcommand*\listoflistings{\listof{codelisting}{List of Listings}}
\makeatother
\makeatletter
\makeatother
\makeatletter
\@ifpackageloaded{caption}{}{\usepackage{caption}}
\@ifpackageloaded{subcaption}{}{\usepackage{subcaption}}
\makeatother

\usepackage{bookmark}

\IfFileExists{xurl.sty}{\usepackage{xurl}}{} % add URL line breaks if available
\urlstyle{same} % disable monospaced font for URLs
\hypersetup{
  pdftitle={PSTAT 5A Practice Worksheet 3},
  pdfauthor={Student Name: \_\_\_\_\_\_\_\_\_\_\_\_\_\_\_\_\_\_\_\_\_\_\_\_},
  colorlinks=true,
  linkcolor={blue},
  filecolor={Maroon},
  citecolor={Blue},
  urlcolor={Blue},
  pdfcreator={LaTeX via pandoc}}


\title{PSTAT 5A Practice Worksheet 3}
\usepackage{etoolbox}
\makeatletter
\providecommand{\subtitle}[1]{% add subtitle to \maketitle
  \apptocmd{\@title}{\par {\large #1 \par}}{}{}
}
\makeatother
\subtitle{Comprehensive Review: Probability, Counting, an Conditional
Probability}
\author{Student Name: \_\_\_\_\_\_\_\_\_\_\_\_\_\_\_\_\_\_\_\_\_\_\_\_}
\date{2025-07-08}

\begin{document}
\maketitle

\renewcommand*\contentsname{Table of contents}
{
\hypersetup{linkcolor=}
\setcounter{tocdepth}{3}
\tableofcontents
}

\section{Instructions and Overview}\label{instructions-and-overview}

\textbf{⏰ Time Allocation:}

\begin{itemize}
\item
  \textbf{Section A (Warm-up):} 8 minutes
\item
  \textbf{Section B (Intermediate):} 15 minutes
\item
  \textbf{Section C (Advanced):} 15 minutes
\item
  \textbf{Section D (Review):} 12 minutes
\item
  \textbf{Total:} 50 minutes
\end{itemize}

\textbf{📝 Important Instructions:}

\begin{itemize}
\item
  Use the formulas provided for guidance
\item
  Round final answers to 4 decimal places unless otherwise specified
\item
  Identify your approach before calculating
\item
  Use calculator as needed
\end{itemize}

\textbf{📚 Key Formulas Reference:}

\textbf{Basic Probability:}

\begin{itemize}
\item
  \textbf{Conditional Probability:}
  \(P(A|B) = \frac{P(A \cap B)}{P(B)}\)
\item
  \textbf{Bayes' Theorem:} \(P(A|B) = \frac{P(B|A) \cdot P(A)}{P(B)}\)
\item
  \textbf{Law of Total Probability:}
  \(P(A) = \sum P(A|B_i) \cdot P(B_i)\)
\item
  \textbf{Addition Rule:} \(P(A \cup B) = P(A) + P(B) - P(A \cap B)\)
\item
  \textbf{Multiplication Rule:}
  \(P(A \cap B) = P(A) \cdot P(B|A) = P(B) \cdot P(A|B)\)
\end{itemize}

\textbf{Counting:}

\begin{itemize}
\item
  \textbf{Permutations:} \(P(n,r) = \frac{n!}{(n-r)!}\)
\item
  \textbf{Combinations:} \(C(n,r) = \binom{n}{r} = \frac{n!}{r!(n-r)!}\)
\end{itemize}

\section{Section A: Probability}\label{section-a-probability}

\emph{⏱️ Estimated time: 8 minutes}

\textbf{Problem A1: Probability Distributions}

Each row in the table below is a proposed grade distribution for a
class. Identify each as a valid or invalid probability distribution, and
explain your reasoning.

\begin{longtable}[]{@{}llllll@{}}
\toprule\noalign{}
Class & A & B & C & D & F \\
\midrule\noalign{}
\endhead
\bottomrule\noalign{}
\endlastfoot
(a) & 0.3 & 0.3 & 0.3 & 0.2 & 0.1 \\
(b) & 0 & 0 & 1 & 0 & 0 \\
(c) & 0.3 & 0.3 & 0.3 & 0 & 0 \\
(d) & 0.3 & 0.5 & 0.2 & 0.1 & -0.1 \\
(e) & 0.2 & 0.4 & 0.2 & 0.1 & 0.1 \\
(f) & 0 & -0.1 & 1.1 & 0 & 0 \\
\end{longtable}

\textbf{Work Space:}

\section{Section B: Permutations and
Combination}\label{section-b-permutations-and-combination}

\emph{⏱️ Estimated time: 15 minutes}

\textbf{Problem B1: Permutations and Combinations}

A cybersecurity team needs to create a secure access protocol.

\textbf{Part (a):} How many 6-character passwords can be formed using 3
specific letters and 3 specific digits if repetitions are not allowed
and letters must come before digits?

\begin{tcolorbox}[enhanced jigsaw, rightrule=.15mm, bottomtitle=1mm, opacitybacktitle=0.6, arc=.35mm, coltitle=black, colbacktitle=quarto-callout-tip-color!10!white, bottomrule=.15mm, colback=white, title=\textcolor{quarto-callout-tip-color}{\faLightbulb}\hspace{0.5em}{Tip}, breakable, leftrule=.75mm, left=2mm, colframe=quarto-callout-tip-color-frame, opacityback=0, toprule=.15mm, toptitle=1mm, titlerule=0mm]

Since letters must come before digits, think of this as two separate
arrangement problems:

\begin{itemize}
\item
  First, arrange the 3 letters in the first 3 positions
\item
  Then, arrange the 3 digits in the last 3 positions
\item
  Use the multiplication principle to combine these results
\end{itemize}

\end{tcolorbox}

\textbf{Part (b):} If the team wants to select 4 people from 12
employees to form a security committee where order doesn't matter, how
many ways can this be done?

\begin{tcolorbox}[enhanced jigsaw, rightrule=.15mm, bottomtitle=1mm, opacitybacktitle=0.6, arc=.35mm, coltitle=black, colbacktitle=quarto-callout-tip-color!10!white, bottomrule=.15mm, colback=white, title=\textcolor{quarto-callout-tip-color}{\faLightbulb}\hspace{0.5em}{Tip}, breakable, leftrule=.75mm, left=2mm, colframe=quarto-callout-tip-color-frame, opacityback=0, toprule=.15mm, toptitle=1mm, titlerule=0mm]

Since order doesn't matter, this is a combination problem. Ask yourself:

\begin{itemize}
\item
  Are we arranging people in specific positions, or just selecting a
  group?
\item
  Which formula should you use: \(P(n,r)\) or \(C(n,r)\)?
\end{itemize}

\end{tcolorbox}

\textbf{Work Space:}

\section{Section C: Conditional
Probability}\label{section-c-conditional-probability}

\emph{⏱️ Estimated time: 15 minutes}

\textbf{Problem B1: Conditional Probability and Medical Testing}

A new COVID variant test has the following characteristics:

\begin{itemize}
\item
  The variant affects 3\% of the tested population
\item
  The test correctly identifies 95\% of people with the variant
  (sensitivity)
\item
  The test correctly identifies 92\% of people without the variant
  (specificity)
\end{itemize}

\textbf{Part (a):} What is the probability that a randomly selected
person tests positive?

\textbf{Part (b):} If someone tests positive, what is the probability
they actually have the variant?

\textbf{Part (c):} If someone tests negative, what is the probability
they actually don't have the variant?

\textbf{Part (d) {[}Challenge{]}:} The health department wants to reduce
false positives. They decide to require two consecutive positive tests
for a positive diagnosis. Assuming test results are independent, what is
the new probability that someone with two positive tests actually has
the variant?

\textbf{Work Space:}

\section{Section C: Conditional
Probability}\label{section-c-conditional-probability-1}

\emph{⏱️ Estimated time: 15 minutes}

\textbf{Problem C1: Advanced Counting with Restrictions}

A restaurant offers a prix fixe menu where customers must choose:

\begin{itemize}
\item
  1 appetizer from 6 options
\item
  1 main course from 8 options
\item
  1 dessert from 5 options
\end{itemize}

However, there are restrictions:

\begin{itemize}
\item
  If you choose the seafood appetizer, you cannot choose the vegetarian
  main course
\item
  If you choose the chocolate dessert, you must choose either the beef
  or chicken main course (3 of the 8 main courses)
\end{itemize}

\textbf{Part (a):} How many valid meal combinations are possible?

\textbf{Part (b):} If customers choose randomly among valid
combinations, what is the probability someone chooses the chocolate
dessert?

\textbf{Work Space:}

\section{Section D: Review}\label{section-d-review}

\emph{⏱️ Estimated time: 12 minutes}

\textbf{Problem B3: Daily Expenses}

Sally gets a cup of coffee and a muffin every day for breakfast from one
of the many coffee shops in her neighborhood. She picks a coffee shop
each morning at random and independently of previous days. The average
price of a cup of coffee is \$1.40 with a standard deviation of 30¢
(\$0.30), the average price of a muffin is \$2.50 with a standard
deviation of 15¢, and the two prices are independent of each other.

\textbf{Part (a):} What is the mean and standard deviation of the amount
she spends on breakfast daily?

\textbf{Part (b):} What is the mean and standard deviation of the amount
she spends on breakfast weekly (7 days)?

\textbf{Work Space:}




\end{document}
