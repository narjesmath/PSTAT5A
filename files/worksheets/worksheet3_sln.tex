% Options for packages loaded elsewhere
\PassOptionsToPackage{unicode}{hyperref}
\PassOptionsToPackage{hyphens}{url}
\PassOptionsToPackage{dvipsnames,svgnames,x11names}{xcolor}
%
\documentclass[
  11pt,
]{article}

\usepackage{amsmath,amssymb}
\usepackage{iftex}
\ifPDFTeX
  \usepackage[T1]{fontenc}
  \usepackage[utf8]{inputenc}
  \usepackage{textcomp} % provide euro and other symbols
\else % if luatex or xetex
  \usepackage{unicode-math}
  \defaultfontfeatures{Scale=MatchLowercase}
  \defaultfontfeatures[\rmfamily]{Ligatures=TeX,Scale=1}
\fi
\usepackage{lmodern}
\ifPDFTeX\else  
    % xetex/luatex font selection
\fi
% Use upquote if available, for straight quotes in verbatim environments
\IfFileExists{upquote.sty}{\usepackage{upquote}}{}
\IfFileExists{microtype.sty}{% use microtype if available
  \usepackage[]{microtype}
  \UseMicrotypeSet[protrusion]{basicmath} % disable protrusion for tt fonts
}{}
\makeatletter
\@ifundefined{KOMAClassName}{% if non-KOMA class
  \IfFileExists{parskip.sty}{%
    \usepackage{parskip}
  }{% else
    \setlength{\parindent}{0pt}
    \setlength{\parskip}{6pt plus 2pt minus 1pt}}
}{% if KOMA class
  \KOMAoptions{parskip=half}}
\makeatother
\usepackage{xcolor}
\usepackage[margin=1in]{geometry}
\setlength{\emergencystretch}{3em} % prevent overfull lines
\setcounter{secnumdepth}{5}
% Make \paragraph and \subparagraph free-standing
\makeatletter
\ifx\paragraph\undefined\else
  \let\oldparagraph\paragraph
  \renewcommand{\paragraph}{
    \@ifstar
      \xxxParagraphStar
      \xxxParagraphNoStar
  }
  \newcommand{\xxxParagraphStar}[1]{\oldparagraph*{#1}\mbox{}}
  \newcommand{\xxxParagraphNoStar}[1]{\oldparagraph{#1}\mbox{}}
\fi
\ifx\subparagraph\undefined\else
  \let\oldsubparagraph\subparagraph
  \renewcommand{\subparagraph}{
    \@ifstar
      \xxxSubParagraphStar
      \xxxSubParagraphNoStar
  }
  \newcommand{\xxxSubParagraphStar}[1]{\oldsubparagraph*{#1}\mbox{}}
  \newcommand{\xxxSubParagraphNoStar}[1]{\oldsubparagraph{#1}\mbox{}}
\fi
\makeatother


\providecommand{\tightlist}{%
  \setlength{\itemsep}{0pt}\setlength{\parskip}{0pt}}\usepackage{longtable,booktabs,array}
\usepackage{calc} % for calculating minipage widths
% Correct order of tables after \paragraph or \subparagraph
\usepackage{etoolbox}
\makeatletter
\patchcmd\longtable{\par}{\if@noskipsec\mbox{}\fi\par}{}{}
\makeatother
% Allow footnotes in longtable head/foot
\IfFileExists{footnotehyper.sty}{\usepackage{footnotehyper}}{\usepackage{footnote}}
\makesavenoteenv{longtable}
\usepackage{graphicx}
\makeatletter
\newsavebox\pandoc@box
\newcommand*\pandocbounded[1]{% scales image to fit in text height/width
  \sbox\pandoc@box{#1}%
  \Gscale@div\@tempa{\textheight}{\dimexpr\ht\pandoc@box+\dp\pandoc@box\relax}%
  \Gscale@div\@tempb{\linewidth}{\wd\pandoc@box}%
  \ifdim\@tempb\p@<\@tempa\p@\let\@tempa\@tempb\fi% select the smaller of both
  \ifdim\@tempa\p@<\p@\scalebox{\@tempa}{\usebox\pandoc@box}%
  \else\usebox{\pandoc@box}%
  \fi%
}
% Set default figure placement to htbp
\def\fps@figure{htbp}
\makeatother

\makeatletter
\@ifpackageloaded{caption}{}{\usepackage{caption}}
\AtBeginDocument{%
\ifdefined\contentsname
  \renewcommand*\contentsname{Table of contents}
\else
  \newcommand\contentsname{Table of contents}
\fi
\ifdefined\listfigurename
  \renewcommand*\listfigurename{List of Figures}
\else
  \newcommand\listfigurename{List of Figures}
\fi
\ifdefined\listtablename
  \renewcommand*\listtablename{List of Tables}
\else
  \newcommand\listtablename{List of Tables}
\fi
\ifdefined\figurename
  \renewcommand*\figurename{Figure}
\else
  \newcommand\figurename{Figure}
\fi
\ifdefined\tablename
  \renewcommand*\tablename{Table}
\else
  \newcommand\tablename{Table}
\fi
}
\@ifpackageloaded{float}{}{\usepackage{float}}
\floatstyle{ruled}
\@ifundefined{c@chapter}{\newfloat{codelisting}{h}{lop}}{\newfloat{codelisting}{h}{lop}[chapter]}
\floatname{codelisting}{Listing}
\newcommand*\listoflistings{\listof{codelisting}{List of Listings}}
\makeatother
\makeatletter
\makeatother
\makeatletter
\@ifpackageloaded{caption}{}{\usepackage{caption}}
\@ifpackageloaded{subcaption}{}{\usepackage{subcaption}}
\makeatother

\usepackage{bookmark}

\IfFileExists{xurl.sty}{\usepackage{xurl}}{} % add URL line breaks if available
\urlstyle{same} % disable monospaced font for URLs
\hypersetup{
  pdftitle={PSTAT 5A Practice Worksheet 3 - SOLUTIONS},
  pdfauthor={Solution Key},
  colorlinks=true,
  linkcolor={blue},
  filecolor={Maroon},
  citecolor={Blue},
  urlcolor={Blue},
  pdfcreator={LaTeX via pandoc}}


\title{PSTAT 5A Practice Worksheet 3 - SOLUTIONS}
\usepackage{etoolbox}
\makeatletter
\providecommand{\subtitle}[1]{% add subtitle to \maketitle
  \apptocmd{\@title}{\par {\large #1 \par}}{}{}
}
\makeatother
\subtitle{Comprehensive Review: Probability, Counting, and Conditional
Probability}
\author{Solution Key}
\date{2025-07-08}

\begin{document}
\maketitle

\renewcommand*\contentsname{Table of contents}
{
\hypersetup{linkcolor=}
\setcounter{tocdepth}{3}
\tableofcontents
}

\section{Section A: Probability -
SOLUTIONS}\label{section-a-probability---solutions}

\emph{⏱️ Estimated time: 8 minutes}

\textbf{Problem A1: Probability Distributions - SOLUTION}

For a valid probability distribution, two conditions must be met:

\begin{enumerate}
\def\labelenumi{\arabic{enumi}.}
\item
  All probabilities must be non-negative (≥ 0)
\item
  The sum of all probabilities must equal 1
\end{enumerate}

\textbf{Analysis:}

\textbf{(a) Invalid}

\begin{itemize}
\tightlist
\item
  Sum = 0.3 + 0.3 + 0.3 + 0.2 + 0.1 = 1.2 \textgreater{} 1 The
  probabilities sum to more than 1, violating the second condition.
\end{itemize}

\textbf{(b) Valid}

\begin{itemize}
\tightlist
\item
  Sum = 0 + 0 + 1 + 0 + 0 = 1 All probabilities are non-negative and sum
  to 1. This represents a class where everyone receives a C.
\end{itemize}

\textbf{(c) Invalid}

\begin{itemize}
\tightlist
\item
  Sum = 0.3 + 0.3 + 0.3 + 0 + 0 = 0.9 \textless{} 1 The probabilities
  sum to less than 1, violating the second condition.
\end{itemize}

\textbf{(d) Invalid}

\begin{itemize}
\tightlist
\item
  Contains F = -0.1 \textless{} 0 Although the sum would equal 1.0, the
  probability for grade F is negative, violating the first condition.
\end{itemize}

\textbf{(e) Valid}

\begin{itemize}
\tightlist
\item
  Sum = 0.2 + 0.4 + 0.2 + 0.1 + 0.1 = 1.0 All probabilities are
  non-negative and sum to 1.
\end{itemize}

\textbf{(f) Invalid}

\begin{itemize}
\tightlist
\item
  Contains B = -0.1 \textless{} 0 Although the sum equals 1.0, the
  probability for grade B is negative, violating the first condition.
\end{itemize}

\section{Section B: Permutations and Combinations -
SOLUTIONS}\label{section-b-permutations-and-combinations---solutions}

\emph{⏱️ Estimated time: 15 minutes}

\textbf{Problem B1: Permutations and Combinations - SOLUTION}

\textbf{Part (a):} How many 6-character passwords can be formed using 3
specific letters and 3 specific digits if repetitions are not allowed
and letters must come before digits?

\textbf{Solution:} Since letters must come before digits, we have a
fixed structure: LLL DDD

\begin{itemize}
\tightlist
\item
  Step 1: Arrange 3 letters in the first 3 positions

  \begin{itemize}
  \tightlist
  \item
    This is a permutation: P(3,3) = 3! = 6 ways
  \end{itemize}
\item
  Step 2: Arrange 3 digits in the last 3 positions

  \begin{itemize}
  \tightlist
  \item
    This is a permutation: P(3,3) = 3! = 6 ways
  \end{itemize}
\item
  Step 3: Apply multiplication principle

  \begin{itemize}
  \tightlist
  \item
    Total passwords = 6 × 6 = \textbf{36 passwords}
  \end{itemize}
\end{itemize}

\textbf{Part (b):} If the team wants to select 4 people from 12
employees to form a security committee where order doesn't matter, how
many ways can this be done?

\textbf{Solution:} Since order doesn't matter, this is a combination
problem.

\[C(12,4) = \binom{12}{4} = \frac{12!}{4!(12-4)!} = \frac{12!}{4! \cdot 8!}\]

\[= \frac{12 \times 11 \times 10 \times 9}{4 \times 3 \times 2 \times 1} = \frac{11880}{24} = \textbf{495 ways}\]

\section{Section C: Conditional Probability -
SOLUTIONS}\label{section-c-conditional-probability---solutions}

\emph{⏱️ Estimated time: 15 minutes}

\textbf{Problem B1: Conditional Probability and Medical Testing -
SOLUTION}

\textbf{Given Information:}

\begin{itemize}
\item
  P(has variant) = 0.03
\item
  P(test positive \textbar{} has variant) = 0.95 (sensitivity)
\item
  P(test negative \textbar{} no variant) = 0.92 (specificity)
\item
  Therefore: P(test positive \textbar{} no variant) = 1 - 0.92 = 0.08
\end{itemize}

\textbf{Part (a):} What is the probability that a randomly selected
person tests positive?

\textbf{Solution:}

Using the Law of Total Probability:

\[P(\text{test positive}) = P(\text{test positive | has variant}) \times P(\text{has variant}) + P(\text{test positive | no variant}) \times P(\text{no variant})\]

\[P(\text{test positive}) = 0.95 \times 0.03 + 0.08 \times 0.97\]
\[= 0.0285 + 0.0776 = \textbf{0.1061}\]

\textbf{Part (b):} If someone tests positive, what is the probability
they actually have the variant?

\textbf{Solution:} Using Bayes' Theorem:

\[P(\text{has variant | test positive}) = \frac{P(\text{test positive | has variant}) \times P(\text{has variant})}{P(\text{test positive})}\]

\[= \frac{0.95 \times 0.03}{0.1061} = \frac{0.0285}{0.1061} = \textbf{0.2686}\]

\textbf{Part (c):} If someone tests negative, what is the probability
they actually don't have the variant?

\textbf{Solution:} First, find P(test negative):
\[P(\text{test negative}) = 1 - P(\text{test positive}) = 1 - 0.1061 = 0.8939\]

Using Bayes' Theorem:
\[P(\text{no variant | test negative}) = \frac{P(\text{test negative | no variant}) \times P(\text{no variant})}{P(\text{test negative})}\]

\[= \frac{0.92 \times 0.97}{0.8939} = \frac{0.8924}{0.8939} = \textbf{0.9983}\]

\textbf{Part (d) {[}Challenge{]}:} Two consecutive positive tests - what
is the probability they actually have the variant?

\textbf{Solution:} Assuming independence between tests:

\[P(\text{two positive | has variant}) = 0.95^2 = 0.9025\]
\[P(\text{two positive | no variant}) = 0.08^2 = 0.0064\]

\[P(\text{two positive}) = 0.9025 \times 0.03 + 0.0064 \times 0.97 = 0.027075 + 0.006208 = 0.033283\]

\[P(\text{has variant | two positive}) = \frac{0.027075}{0.033283} = \textbf{0.8134}\]

\textbf{Problem C1: Advanced Counting with Restrictions - SOLUTION}

\textbf{Part (a):} How many valid meal combinations are possible?

\textbf{Solution:} We need to consider cases based on the restrictions.

\textbf{Case 1: Seafood appetizer is chosen}

\begin{itemize}
\item
  1 appetizer option (seafood)
\item
  7 main course options (cannot choose vegetarian)
\item
  5 dessert options
\item
  Combinations: 1 × 7 × 5 = 35
\end{itemize}

\textbf{Case 2: Non-seafood appetizer + chocolate dessert}

\begin{itemize}
\item
  5 appetizer options (non-seafood)
\item
  3 main course options (only beef or chicken allowed with chocolate)
\item
  1 dessert option (chocolate)
\item
  Combinations: 5 × 3 × 1 = 15
\end{itemize}

\textbf{Case 3: Non-seafood appetizer + non-chocolate dessert} - 5
appetizer options (non-seafood)

\begin{itemize}
\item
  8 main course options (no restrictions)
\item
  4 dessert options (non-chocolate)
\item
  Combinations: 5 × 8 × 4 = 160
\end{itemize}

\textbf{Total valid combinations:} 35 + 15 + 160 = \textbf{210
combinations}

\textbf{Part (b):} If customers choose randomly among valid
combinations, what is the probability someone chooses the chocolate
dessert?

\textbf{Solution:} Combinations with chocolate dessert: 15 (from Case 2
above) Total valid combinations: 210

\[P(\text{chocolate dessert}) = \frac{15}{210} = \frac{1}{14} = \textbf{0.0714}\]

\section{Section D: Review -
SOLUTIONS}\label{section-d-review---solutions}

\emph{⏱️ Estimated time: 12 minutes}

\textbf{Problem B3: Daily Expenses - SOLUTION}

\textbf{Given:}

\begin{itemize}
\item
  Coffee: Mean = \$1.40, SD = \$0.30
\item
  Muffin: Mean = \$2.50, SD = \$0.15
\item
  Prices are independent
\end{itemize}

\textbf{Part (a):} What is the mean and standard deviation of the amount
she spends on breakfast daily?

\textbf{Solution:} For the sum of independent random variables:

\textbf{Mean of daily expenses:}
\[E[\text{Daily}] = E[\text{Coffee}] + E[\text{Muffin}] = \$1.40 + \$2.50 = \textbf{\$3.90}\]

\textbf{Variance of daily expenses:}
\[\text{Var}[\text{Daily}] = \text{Var}[\text{Coffee}] + \text{Var}[\text{Muffin}] = (0.30)^2 + (0.15)^2 = 0.09 + 0.0225 = 0.1125\]

\textbf{Standard deviation of daily expenses:}
\[SD[\text{Daily}] = \sqrt{0.1125} = \textbf{\$0.3354}\]

\textbf{Part (b):} What is the mean and standard deviation of the amount
she spends on breakfast weekly (7 days)?

\textbf{Solution:} For the sum of 7 independent daily expenses:

\textbf{Mean of weekly expenses:}
\[E[\text{Weekly}] = 7 \times E[\text{Daily}] = 7 \times \$3.90 = \textbf{\$27.30}\]

\textbf{Variance of weekly expenses:}
\[\text{Var}[\text{Weekly}] = 7 \times \text{Var}[\text{Daily}] = 7 \times 0.1125 = 0.7875\]

\textbf{Standard deviation of weekly expenses:}
\[SD[\text{Weekly}] = \sqrt{0.7875} = \textbf{\$0.8874}\]




\end{document}
