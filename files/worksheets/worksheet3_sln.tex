% Options for packages loaded elsewhere
\PassOptionsToPackage{unicode}{hyperref}
\PassOptionsToPackage{hyphens}{url}
\PassOptionsToPackage{dvipsnames,svgnames,x11names}{xcolor}
%
\documentclass[
  11pt,
]{article}

\usepackage{amsmath,amssymb}
\usepackage{iftex}
\ifPDFTeX
  \usepackage[T1]{fontenc}
  \usepackage[utf8]{inputenc}
  \usepackage{textcomp} % provide euro and other symbols
\else % if luatex or xetex
  \usepackage{unicode-math}
  \defaultfontfeatures{Scale=MatchLowercase}
  \defaultfontfeatures[\rmfamily]{Ligatures=TeX,Scale=1}
\fi
\usepackage{lmodern}
\ifPDFTeX\else  
    % xetex/luatex font selection
\fi
% Use upquote if available, for straight quotes in verbatim environments
\IfFileExists{upquote.sty}{\usepackage{upquote}}{}
\IfFileExists{microtype.sty}{% use microtype if available
  \usepackage[]{microtype}
  \UseMicrotypeSet[protrusion]{basicmath} % disable protrusion for tt fonts
}{}
\makeatletter
\@ifundefined{KOMAClassName}{% if non-KOMA class
  \IfFileExists{parskip.sty}{%
    \usepackage{parskip}
  }{% else
    \setlength{\parindent}{0pt}
    \setlength{\parskip}{6pt plus 2pt minus 1pt}}
}{% if KOMA class
  \KOMAoptions{parskip=half}}
\makeatother
\usepackage{xcolor}
\usepackage[margin=1in]{geometry}
\setlength{\emergencystretch}{3em} % prevent overfull lines
\setcounter{secnumdepth}{5}
% Make \paragraph and \subparagraph free-standing
\makeatletter
\ifx\paragraph\undefined\else
  \let\oldparagraph\paragraph
  \renewcommand{\paragraph}{
    \@ifstar
      \xxxParagraphStar
      \xxxParagraphNoStar
  }
  \newcommand{\xxxParagraphStar}[1]{\oldparagraph*{#1}\mbox{}}
  \newcommand{\xxxParagraphNoStar}[1]{\oldparagraph{#1}\mbox{}}
\fi
\ifx\subparagraph\undefined\else
  \let\oldsubparagraph\subparagraph
  \renewcommand{\subparagraph}{
    \@ifstar
      \xxxSubParagraphStar
      \xxxSubParagraphNoStar
  }
  \newcommand{\xxxSubParagraphStar}[1]{\oldsubparagraph*{#1}\mbox{}}
  \newcommand{\xxxSubParagraphNoStar}[1]{\oldsubparagraph{#1}\mbox{}}
\fi
\makeatother


\providecommand{\tightlist}{%
  \setlength{\itemsep}{0pt}\setlength{\parskip}{0pt}}\usepackage{longtable,booktabs,array}
\usepackage{calc} % for calculating minipage widths
% Correct order of tables after \paragraph or \subparagraph
\usepackage{etoolbox}
\makeatletter
\patchcmd\longtable{\par}{\if@noskipsec\mbox{}\fi\par}{}{}
\makeatother
% Allow footnotes in longtable head/foot
\IfFileExists{footnotehyper.sty}{\usepackage{footnotehyper}}{\usepackage{footnote}}
\makesavenoteenv{longtable}
\usepackage{graphicx}
\makeatletter
\newsavebox\pandoc@box
\newcommand*\pandocbounded[1]{% scales image to fit in text height/width
  \sbox\pandoc@box{#1}%
  \Gscale@div\@tempa{\textheight}{\dimexpr\ht\pandoc@box+\dp\pandoc@box\relax}%
  \Gscale@div\@tempb{\linewidth}{\wd\pandoc@box}%
  \ifdim\@tempb\p@<\@tempa\p@\let\@tempa\@tempb\fi% select the smaller of both
  \ifdim\@tempa\p@<\p@\scalebox{\@tempa}{\usebox\pandoc@box}%
  \else\usebox{\pandoc@box}%
  \fi%
}
% Set default figure placement to htbp
\def\fps@figure{htbp}
\makeatother

\makeatletter
\@ifpackageloaded{caption}{}{\usepackage{caption}}
\AtBeginDocument{%
\ifdefined\contentsname
  \renewcommand*\contentsname{Table of contents}
\else
  \newcommand\contentsname{Table of contents}
\fi
\ifdefined\listfigurename
  \renewcommand*\listfigurename{List of Figures}
\else
  \newcommand\listfigurename{List of Figures}
\fi
\ifdefined\listtablename
  \renewcommand*\listtablename{List of Tables}
\else
  \newcommand\listtablename{List of Tables}
\fi
\ifdefined\figurename
  \renewcommand*\figurename{Figure}
\else
  \newcommand\figurename{Figure}
\fi
\ifdefined\tablename
  \renewcommand*\tablename{Table}
\else
  \newcommand\tablename{Table}
\fi
}
\@ifpackageloaded{float}{}{\usepackage{float}}
\floatstyle{ruled}
\@ifundefined{c@chapter}{\newfloat{codelisting}{h}{lop}}{\newfloat{codelisting}{h}{lop}[chapter]}
\floatname{codelisting}{Listing}
\newcommand*\listoflistings{\listof{codelisting}{List of Listings}}
\makeatother
\makeatletter
\makeatother
\makeatletter
\@ifpackageloaded{caption}{}{\usepackage{caption}}
\@ifpackageloaded{subcaption}{}{\usepackage{subcaption}}
\makeatother

\usepackage{bookmark}

\IfFileExists{xurl.sty}{\usepackage{xurl}}{} % add URL line breaks if available
\urlstyle{same} % disable monospaced font for URLs
\hypersetup{
  pdftitle={PSTAT 5A Practice Worksheet 3 - SOLUTIONS},
  pdfauthor={Narjes Mathlouthi},
  colorlinks=true,
  linkcolor={blue},
  filecolor={Maroon},
  citecolor={Blue},
  urlcolor={Blue},
  pdfcreator={LaTeX via pandoc}}


\title{PSTAT 5A Practice Worksheet 3 - SOLUTIONS}
\usepackage{etoolbox}
\makeatletter
\providecommand{\subtitle}[1]{% add subtitle to \maketitle
  \apptocmd{\@title}{\par {\large #1 \par}}{}{}
}
\makeatother
\subtitle{Comprehensive Review: Probability, Counting, and Conditional
Probability}
\author{Narjes Mathlouthi}
\date{2025-07-09}

\begin{document}
\maketitle

\renewcommand*\contentsname{Table of contents}
{
\hypersetup{linkcolor=}
\setcounter{tocdepth}{3}
\tableofcontents
}

\textbf{📚 Key Formulas Reference:}

\textbf{Basic Probability:}

\begin{itemize}
\item
  \textbf{Conditional Probability:}
  \(P(A|B) = \frac{P(A \cap B)}{P(B)}\)
\item
  \textbf{Law of Total Probability:}
  \(P(A) = \sum P(A|B_i) \cdot P(B_i)\)
\item
  \textbf{Addition Rule:} \(P(A \cup B) = P(A) + P(B) - P(A \cap B)\)
\item
  \textbf{Multiplication Rule:}
  \(P(A \cap B) = P(A) \cdot P(B|A) = P(B) \cdot P(A|B)\)
\end{itemize}

\textbf{Counting:}

\begin{itemize}
\item
  \textbf{Multiplication Rule:} If a procedure consists of \(k\) steps,
  with \(n_1\) ways for step 1, \(n_2\) for step 2, \ldots, \(n_k\) for
  step \(k\), then total ways:
  \(n_1 \times n_2 \times \cdots \times n_k\)
\item
  \textbf{Factorial:}
  \(n! = n \times (n-1) \times (n-2) \times \cdots \times 2 \times 1\)
\item
  \textbf{Permutations:} \(P(n,r) = \frac{n!}{(n-r)!}\)
\item
  \textbf{Combinations:} \(C(n,r) = \binom{n}{r} = \frac{n!}{r!(n-r)!}\)
\end{itemize}

\section{Section A: Probability -
SOLUTIONS}\label{section-a-probability---solutions}

\emph{⏱️ Estimated time: 8 minutes}

\textbf{Problem A1: Probability Distributions - SOLUTION}

For a valid probability distribution, two conditions must be met:

\begin{enumerate}
\def\labelenumi{\arabic{enumi}.}
\item
  All probabilities must be non-negative (≥ 0)
\item
  The sum of all probabilities must equal 1
\end{enumerate}

\textbf{Analysis:}

\textbf{(a) Invalid}
\[\text{Sum} = 0.3 + 0.3 + 0.3 + 0.2 + 0.1 = 1.2 > 1\] The probabilities
sum to more than 1, violating condition 2.

\textbf{(b) Valid} \[\text{Sum} = 0 + 0 + 1 + 0 + 0 = 1\] All
probabilities are non-negative and sum to 1. This represents a class
where everyone receives a C.

\textbf{(c) Invalid} \[\text{Sum} = 0.3 + 0.3 + 0.3 + 0 + 0 = 0.9 < 1\]
The probabilities sum to less than 1, violating condition 2.

\textbf{(d) Invalid} Contains \(P(F) = -0.1 < 0\) Although the sum would
equal 1.0, the probability for grade F is negative, violating condition
1.

\textbf{(e) Valid} \[\text{Sum} = 0.2 + 0.4 + 0.2 + 0.1 + 0.1 = 1.0\]
All probabilities are non-negative and sum to 1.

\textbf{(f) Invalid} Contains \(P(B) = -0.1 < 0\) Although the sum
equals 1.0, the probability for grade B is negative, violating condition
1.

\section{Section B: Permutations and Combinations -
SOLUTIONS}\label{section-b-permutations-and-combinations---solutions}

\emph{⏱️ Estimated time: 15 minutes}

\textbf{Problem B1: Permutations and Combinations - SOLUTION}

\textbf{Part (a):} How many 6-character passwords can be formed using 3
specific letters and 3 specific digits if repetitions are not allowed
and letters must come before digits?

\textbf{Solution:} Since letters must come before digits, we have a
fixed structure: \textbf{LLL DDD}

\textbf{Step 1:} Arrange 3 letters in the first 3 positions

\begin{itemize}
\tightlist
\item
  This is a permutation:
  \(P(3,3) = \frac{3!}{(3-3)!} = \frac{3!}{0!} = 3! = 6\) ways
\end{itemize}

\textbf{Step 2:} Arrange 3 digits in the last 3 positions

\begin{itemize}
\tightlist
\item
  This is a permutation:
  \(P(3,3) = \frac{3!}{(3-3)!} = \frac{3!}{0!} = 3! = 6\) ways
\end{itemize}

\textbf{Step 3:} Apply multiplication principle
\[\text{Total passwords} = 6 \times 6 = \boxed{36 \text{ passwords}}\]

\textbf{Part (b):} If the team wants to select 4 people from 12
employees to form a security committee where order doesn't matter, how
many ways can this be done?

\textbf{Solution:} Since order doesn't matter, this is a
\textbf{combination} problem.

\[C(12,4) = \binom{12}{4} = \frac{12!}{4!(12-4)!} = \frac{12!}{4! \cdot 8!}\]

\[= \frac{12 \times 11 \times 10 \times 9}{4 \times 3 \times 2 \times 1} = \frac{11,880}{24} = \boxed{495 \text{ ways}}\]

\section{Section C: Conditional Probability -
SOLUTIONS}\label{section-c-conditional-probability---solutions}

\emph{⏱️ Estimated time: 12 minutes}

\textbf{Problem C1: Drawing Cards (Without Replacement) - SOLUTION}

\textbf{Given Information:}

\begin{itemize}
\item
  Standard 52-card deck
\item
  Drawing two cards without replacement
\item
  \(A = \{\text{"first card is a heart"}\}\)
\item
  \(B = \{\text{"second card is an ace"}\}\)
\end{itemize}

\textbf{Solution:}

\textbf{1. P(A)}

There are 13 hearts in a 52-card deck.
\[P(A) = \frac{13}{52} = \boxed{\frac{1}{4} = 0.2500}\]

\textbf{2. P(A and B)}

We need both events to occur: first card is a heart AND second card is
an ace.

\textbf{Case 1:} First card is the Ace of Hearts -
\(P(\text{1st card is Ace of Hearts}) = \frac{1}{52}\)

\begin{itemize}
\item
  \(P(\text{2nd card is an ace | 1st card is Ace of Hearts}) = \frac{3}{51}\)
  (3 aces left)
\item
  \(P(\text{Case 1}) = \frac{1}{52} \times \frac{3}{51} = \frac{3}{2652}\)
\end{itemize}

\textbf{Case 2:} First card is a non-ace heart -
\(P(\text{1st card is non-ace heart}) = \frac{12}{52}\) (12 non-ace
hearts)

\begin{itemize}
\item
  \(P(\text{2nd card is an ace | 1st card is non-ace heart}) = \frac{4}{51}\)
  (4 aces left)
\item
  \(P(\text{Case 2}) = \frac{12}{52} \times \frac{4}{51} = \frac{48}{2652}\)
\end{itemize}

\[P(A \text{ and } B) = \frac{3}{2652} + \frac{48}{2652} = \frac{51}{2652} = \boxed{\frac{1}{52} = 0.0192}\]

\textbf{3. P(B\textbar A)}

Using the definition of conditional probability:
\[P(B|A) = \frac{P(A \text{ and } B)}{P(A)} = \frac{\frac{1}{52}}{\frac{1}{4}} = \frac{1}{52} \times \frac{4}{1} = \boxed{\frac{4}{52} = \frac{1}{13} = 0.0769}\]

\textbf{Alternative approach:} Given that the first card is a heart:

\begin{itemize}
\item
  If it's the Ace of Hearts: 3 aces remain out of 51 cards
\item
  If it's a non-ace heart: 4 aces remain out of 51 cards
\item
  \(P(B|A) = \frac{1}{13} \times \frac{3}{51} + \frac{12}{13} \times \frac{4}{51} = \frac{3 + 48}{13 \times 51} = \frac{51}{663} = \frac{4}{52} = \frac{1}{13}\)
\end{itemize}

\textbf{4. P(B)}

Using the Law of Total Probability. Let \(A^c\) = ``first card is not a
heart''

\[P(B) = P(B|A) \cdot P(A) + P(B|A^c) \cdot P(A^c)\]

We know:

\begin{itemize}
\item
  \(P(A) = \frac{1}{4}\) and \(P(A^c) = \frac{3}{4}\)
\item
  \(P(B|A) = \frac{1}{13}\) (from part 3)
\item
  \(P(B|A^c) = \frac{4}{51}\) (if first card isn't a heart, all 4 aces
  remain)
\end{itemize}

\[P(B) = \frac{1}{13} \times \frac{1}{4} + \frac{4}{51} \times \frac{3}{4} = \frac{1}{52} + \frac{12}{204} = \frac{1}{52} + \frac{3}{51}\]

\[= \frac{51 + 156}{52 \times 51} = \frac{207}{2652} = \boxed{\frac{4}{51} = 0.0784}\]

\textbf{5. Comparison of P(B\textbar A) vs P(B)}

\[P(B|A) = \frac{1}{13} = 0.0769\] \[P(B) = \frac{4}{51} = 0.0784\]

\textbf{Analysis:} \(P(B|A) < P(B)\)

\textbf{Explanation:} The probability of getting an ace on the second
draw is slightly \textbf{lower} when we know the first card is a heart
compared to when we don't know anything about the first card. This
happens because:

\begin{itemize}
\item
  When the first card is a heart, there's a \(\frac{1}{13}\) chance it's
  the Ace of Hearts, removing one ace from the deck
\item
  This makes it slightly less likely to draw an ace on the second draw
\item
  This demonstrates \textbf{dependence} - the events are not independent
  because drawing without replacement creates dependence between
  successive draws
\end{itemize}

\section{Section D: Advanced Counting with Restrictions -
SOLUTIONS}\label{section-d-advanced-counting-with-restrictions---solutions}

\emph{⏱️ Estimated time: 15 minutes}

\textbf{Problem D1: Advanced Counting with Restrictions - SOLUTION}

\textbf{Given:}

\begin{itemize}
\item
  6 appetizer options (including 1 seafood)
\item
  8 main course options (including 1 vegetarian, and 3 that are beef or
  chicken)
\item
  5 dessert options (including 1 chocolate)
\end{itemize}

\textbf{Restrictions:}

\begin{enumerate}
\def\labelenumi{\arabic{enumi}.}
\item
  Seafood appetizer → cannot choose vegetarian main course
\item
  Chocolate dessert → must choose beef or chicken main course (3
  specific options)
\end{enumerate}

\textbf{Part (a):} How many valid meal combinations are possible?

\textbf{Solution using cases:}

\textbf{Case 1: Seafood appetizer is chosen}

\begin{itemize}
\item
  1 appetizer choice (seafood)
\item
  7 main course choices (8 total minus 1 vegetarian)
\item
  5 dessert choices (no restrictions)
\item
  Total: \(1 \times 7 \times 5 = 35\) combinations
\end{itemize}

\textbf{Case 2: Non-seafood appetizer + chocolate dessert}

\begin{itemize}
\item
  5 appetizer choices (6 total minus 1 seafood)
\item
  3 main course choices (only beef or chicken allowed with chocolate)
\item
  1 dessert choice (chocolate)
\item
  Total: \(5 \times 3 \times 1 = 15\) combinations
\end{itemize}

\textbf{Case 3: Non-seafood appetizer + non-chocolate dessert}

\begin{itemize}
\item
  5 appetizer choices (6 total minus 1 seafood)
\item
  8 main course choices (no restrictions since no seafood appetizer)
\item
  4 dessert choices (5 total minus 1 chocolate)
\item
  Total: \(5 \times 8 \times 4 = 160\) combinations
\end{itemize}

\textbf{Total valid combinations:}
\[35 + 15 + 160 = \boxed{210 \text{ combinations}}\]

\textbf{Verification using complementary counting:}

\begin{itemize}
\item
  Total unrestricted combinations: \(6 \times 8 \times 5 = 240\)
\item
  Invalid combinations to subtract:

  \begin{itemize}
  \item
    Seafood + vegetarian + any dessert: \(1 \times 1 \times 5 = 5\)
  \item
    Non-seafood + chocolate + non-beef/chicken:
    \(5 \times 5 \times 1 = 25\)
  \end{itemize}
\item
  Valid combinations: \(240 - 5 - 25 = 210\) ✓
\end{itemize}

\textbf{Part (b):} If customers choose randomly among valid
combinations, what is the probability someone chooses the chocolate
dessert?

\textbf{Solution:} From our case analysis, combinations with chocolate
dessert come only from Case 2:

\begin{itemize}
\item
  Combinations with chocolate dessert: 15
\item
  Total valid combinations: 210
\end{itemize}

\[P(\text{chocolate dessert}) = \frac{15}{210} = \frac{1}{14} = \boxed{0.0714}\]

\textbf{Alternative verification:}

We can also calculate this directly:

\begin{itemize}
\item
  Non-seafood appetizers: 5 choices
\item
  With chocolate dessert, must choose from 3 main courses
\item
  Valid chocolate combinations: \(5 \times 3 = 15\)
\item
  Probability: \(\frac{15}{210} = \frac{1}{14} = 0.0714\) ✓
\end{itemize}

\section{Summary of Key Concepts}\label{summary-of-key-concepts}

\textbf{Probability Distributions}

\begin{itemize}
\item
  Valid distributions require: all probabilities \(\geq 0\) and sum
  \(= 1\)
\item
  Check both conditions systematically
\end{itemize}

\textbf{Permutations vs Combinations}

\begin{itemize}
\item
  Permutations: Order matters, use \(P(n,r) = \frac{n!}{(n-r)!}\)
\item
  Combinations: Order doesn't matter, use
  \(C(n,r) = \frac{n!}{r!(n-r)!}\)
\item
  Multiplication principle: Combine independent choices
\end{itemize}

\textbf{Conditional Probability}

\begin{itemize}
\item
  Without replacement: Creates dependence between events
\item
  Use definition: \(P(B|A) = \frac{P(A \cap B)}{P(A)}\)
\item
  Law of Total Probability: For calculating marginal probabilities
\end{itemize}

\textbf{Advanced Counting}

\begin{itemize}
\item
  Case analysis: Break complex problems into manageable parts
\item
  Handle restrictions: Consider what's allowed vs.~not allowed
\item
  Verification: Use complementary counting or direct calculation
\end{itemize}




\end{document}
