% Options for packages loaded elsewhere
% Options for packages loaded elsewhere
\PassOptionsToPackage{unicode}{hyperref}
\PassOptionsToPackage{hyphens}{url}
\PassOptionsToPackage{dvipsnames,svgnames,x11names}{xcolor}
%
\documentclass[
  11pt,
]{article}
\usepackage{xcolor}
\usepackage[margin=1in]{geometry}
\usepackage{amsmath,amssymb}
\setcounter{secnumdepth}{5}
\usepackage{iftex}
\ifPDFTeX
  \usepackage[T1]{fontenc}
  \usepackage[utf8]{inputenc}
  \usepackage{textcomp} % provide euro and other symbols
\else % if luatex or xetex
  \usepackage{unicode-math} % this also loads fontspec
  \defaultfontfeatures{Scale=MatchLowercase}
  \defaultfontfeatures[\rmfamily]{Ligatures=TeX,Scale=1}
\fi
\usepackage{lmodern}
\ifPDFTeX\else
  % xetex/luatex font selection
\fi
% Use upquote if available, for straight quotes in verbatim environments
\IfFileExists{upquote.sty}{\usepackage{upquote}}{}
\IfFileExists{microtype.sty}{% use microtype if available
  \usepackage[]{microtype}
  \UseMicrotypeSet[protrusion]{basicmath} % disable protrusion for tt fonts
}{}
\makeatletter
\@ifundefined{KOMAClassName}{% if non-KOMA class
  \IfFileExists{parskip.sty}{%
    \usepackage{parskip}
  }{% else
    \setlength{\parindent}{0pt}
    \setlength{\parskip}{6pt plus 2pt minus 1pt}}
}{% if KOMA class
  \KOMAoptions{parskip=half}}
\makeatother
% Make \paragraph and \subparagraph free-standing
\makeatletter
\ifx\paragraph\undefined\else
  \let\oldparagraph\paragraph
  \renewcommand{\paragraph}{
    \@ifstar
      \xxxParagraphStar
      \xxxParagraphNoStar
  }
  \newcommand{\xxxParagraphStar}[1]{\oldparagraph*{#1}\mbox{}}
  \newcommand{\xxxParagraphNoStar}[1]{\oldparagraph{#1}\mbox{}}
\fi
\ifx\subparagraph\undefined\else
  \let\oldsubparagraph\subparagraph
  \renewcommand{\subparagraph}{
    \@ifstar
      \xxxSubParagraphStar
      \xxxSubParagraphNoStar
  }
  \newcommand{\xxxSubParagraphStar}[1]{\oldsubparagraph*{#1}\mbox{}}
  \newcommand{\xxxSubParagraphNoStar}[1]{\oldsubparagraph{#1}\mbox{}}
\fi
\makeatother


\usepackage{longtable,booktabs,array}
\usepackage{calc} % for calculating minipage widths
% Correct order of tables after \paragraph or \subparagraph
\usepackage{etoolbox}
\makeatletter
\patchcmd\longtable{\par}{\if@noskipsec\mbox{}\fi\par}{}{}
\makeatother
% Allow footnotes in longtable head/foot
\IfFileExists{footnotehyper.sty}{\usepackage{footnotehyper}}{\usepackage{footnote}}
\makesavenoteenv{longtable}
\usepackage{graphicx}
\makeatletter
\newsavebox\pandoc@box
\newcommand*\pandocbounded[1]{% scales image to fit in text height/width
  \sbox\pandoc@box{#1}%
  \Gscale@div\@tempa{\textheight}{\dimexpr\ht\pandoc@box+\dp\pandoc@box\relax}%
  \Gscale@div\@tempb{\linewidth}{\wd\pandoc@box}%
  \ifdim\@tempb\p@<\@tempa\p@\let\@tempa\@tempb\fi% select the smaller of both
  \ifdim\@tempa\p@<\p@\scalebox{\@tempa}{\usebox\pandoc@box}%
  \else\usebox{\pandoc@box}%
  \fi%
}
% Set default figure placement to htbp
\def\fps@figure{htbp}
\makeatother





\setlength{\emergencystretch}{3em} % prevent overfull lines

\providecommand{\tightlist}{%
  \setlength{\itemsep}{0pt}\setlength{\parskip}{0pt}}



 


\makeatletter
\@ifpackageloaded{tcolorbox}{}{\usepackage[skins,breakable]{tcolorbox}}
\@ifpackageloaded{fontawesome5}{}{\usepackage{fontawesome5}}
\definecolor{quarto-callout-color}{HTML}{909090}
\definecolor{quarto-callout-note-color}{HTML}{0758E5}
\definecolor{quarto-callout-important-color}{HTML}{CC1914}
\definecolor{quarto-callout-warning-color}{HTML}{EB9113}
\definecolor{quarto-callout-tip-color}{HTML}{00A047}
\definecolor{quarto-callout-caution-color}{HTML}{FC5300}
\definecolor{quarto-callout-color-frame}{HTML}{acacac}
\definecolor{quarto-callout-note-color-frame}{HTML}{4582ec}
\definecolor{quarto-callout-important-color-frame}{HTML}{d9534f}
\definecolor{quarto-callout-warning-color-frame}{HTML}{f0ad4e}
\definecolor{quarto-callout-tip-color-frame}{HTML}{02b875}
\definecolor{quarto-callout-caution-color-frame}{HTML}{fd7e14}
\makeatother
\makeatletter
\@ifpackageloaded{caption}{}{\usepackage{caption}}
\AtBeginDocument{%
\ifdefined\contentsname
  \renewcommand*\contentsname{Table of contents}
\else
  \newcommand\contentsname{Table of contents}
\fi
\ifdefined\listfigurename
  \renewcommand*\listfigurename{List of Figures}
\else
  \newcommand\listfigurename{List of Figures}
\fi
\ifdefined\listtablename
  \renewcommand*\listtablename{List of Tables}
\else
  \newcommand\listtablename{List of Tables}
\fi
\ifdefined\figurename
  \renewcommand*\figurename{Figure}
\else
  \newcommand\figurename{Figure}
\fi
\ifdefined\tablename
  \renewcommand*\tablename{Table}
\else
  \newcommand\tablename{Table}
\fi
}
\@ifpackageloaded{float}{}{\usepackage{float}}
\floatstyle{ruled}
\@ifundefined{c@chapter}{\newfloat{codelisting}{h}{lop}}{\newfloat{codelisting}{h}{lop}[chapter]}
\floatname{codelisting}{Listing}
\newcommand*\listoflistings{\listof{codelisting}{List of Listings}}
\makeatother
\makeatletter
\makeatother
\makeatletter
\@ifpackageloaded{caption}{}{\usepackage{caption}}
\@ifpackageloaded{subcaption}{}{\usepackage{subcaption}}
\makeatother
\usepackage{bookmark}
\IfFileExists{xurl.sty}{\usepackage{xurl}}{} % add URL line breaks if available
\urlstyle{same}
\hypersetup{
  pdftitle={PSTAT 5A Practice Worksheet 5},
  pdfauthor={Student Name: \_\_\_\_\_\_\_\_\_\_\_\_\_\_\_\_\_\_\_\_\_\_\_\_},
  colorlinks=true,
  linkcolor={blue},
  filecolor={Maroon},
  citecolor={Blue},
  urlcolor={Blue},
  pdfcreator={LaTeX via pandoc}}


\title{PSTAT 5A Practice Worksheet 5}
\usepackage{etoolbox}
\makeatletter
\providecommand{\subtitle}[1]{% add subtitle to \maketitle
  \apptocmd{\@title}{\par {\large #1 \par}}{}{}
}
\makeatother
\subtitle{Continuous Random Variables and Confidence Intervals}
\author{Student Name: \_\_\_\_\_\_\_\_\_\_\_\_\_\_\_\_\_\_\_\_\_\_\_\_}
\date{2025-07-23}
\begin{document}
\maketitle

\renewcommand*\contentsname{Table of contents}
{
\hypersetup{linkcolor=}
\setcounter{tocdepth}{3}
\tableofcontents
}

\section{Instructions and Overview}\label{instructions-and-overview}

\textbf{⏰ Time Allocation:}

\begin{itemize}
\item
  \textbf{Intro \& Setup} : 10 minutes
\item
  \textbf{Section A (Continuous Distributions):} 20 minutes
\item
  \textbf{Section B (Confidence Intervals):} 20 minutes
\item
  \textbf{Optional Questions:} Do on your own
\item
  \textbf{Total:} 50 minutes
\end{itemize}

\textbf{📝 Important Instructions:}

\begin{itemize}
\item
  Use the formulas and tables provided for guidance
\item
  Round final answers to 4 decimal places unless otherwise specified
\item
  For confidence intervals, always interpret your results in context
\item
  Use
  \href{https://math.arizona.edu/~rsims/ma464/standardnormaltable.pdf}{z-table}
  or
  \href{https://www.sjsu.edu/faculty/gerstman/StatPrimer/t-table.pdf}{t-table}
  as appropriate
\item
  Show your work for all calculations
\end{itemize}

\textbf{📚 Key Formulas Reference:}

\textbf{Continuous Random Variables:}

\textbf{Normal Distribution:} \(X \sim N(\mu, \sigma^2)\)

\begin{itemize}
\tightlist
\item
  \textbf{PDF:}
  \(f(x) = \frac{1}{\sigma\sqrt{2\pi}} e^{-\frac{(x-\mu)^2}{2\sigma^2}}\)
\item
  \textbf{Standardization:} \(Z = \frac{X - \mu}{\sigma}\) where
  \(Z \sim N(0,1)\)
\item
  \textbf{Mean:} \(E[X] = \mu\)
\item
  \textbf{Variance:} \(\text{Var}(X) = \sigma^2\)
\end{itemize}

\textbf{Uniform Distribution:} \(X \sim \text{Uniform}(a,b)\)

\begin{itemize}
\tightlist
\item
  \textbf{PDF:} \(f(x) = \frac{1}{b-a}\) for \(a \leq x \leq b\)
\item
  \textbf{Mean:} \(E[X] = \frac{a+b}{2}\)
\item
  \textbf{Variance:} \(\text{Var}(X) = \frac{(b-a)^2}{12}\)
\end{itemize}

\textbf{Exponential Distribution:}
\(X \sim \text{Exponential}(\lambda)\)

\begin{itemize}
\tightlist
\item
  \textbf{PDF:} \(f(x) = \lambda e^{-\lambda x}\) for \(x \geq 0\)
\item
  \textbf{Mean:} \(E[X] = \frac{1}{\lambda}\)
\item
  \textbf{Variance:} \(\text{Var}(X) = \frac{1}{\lambda^2}\)
\end{itemize}

\textbf{Confidence Intervals:}

\textbf{For Population Mean (σ known):}
\(\bar{x} \pm z_{\alpha/2} \cdot \frac{\sigma}{\sqrt{n}}\)

\textbf{For Population Mean (σ unknown):}
\(\bar{x} \pm t_{\alpha/2} \cdot \frac{s}{\sqrt{n}}\)

\textbf{Margin of Error:}
\(E = z_{\alpha/2} \cdot \frac{\sigma}{\sqrt{n}}\) or
\(E = t_{\alpha/2} \cdot \frac{s}{\sqrt{n}}\)

\textbf{Sample Size:}
\(n = \left(\frac{z_{\alpha/2} \cdot \sigma}{E}\right)^2\)

\section{Section A: Continuous Random
Variables}\label{section-a-continuous-random-variables}

\emph{⏱️ Estimated time: 20 minutes}

\textbf{Problem A1: Distribution Identification and Properties}

For each scenario below, identify the appropriate continuous
distribution and find the requested values:

\textbf{(a)} The time (in minutes) between arrivals at a coffee shop
follows an exponential distribution with an average of 2 minutes between
arrivals.

\begin{itemize}
\tightlist
\item
  What is the parameter \(\lambda\)?
\item
  What is the probability that the next customer arrives within 1
  minute?
\end{itemize}

\textbf{(b)} A random number generator produces values uniformly between
10 and 30.

\begin{itemize}
\tightlist
\item
  What are the parameters a and b?
\item
  What is the expected value and variance?
\end{itemize}

\textbf{Work Space:}

\textbf{Problem A2: Normal Distribution Calculations}

The heights of adult women in the US are normally distributed with
\(\mu = 64\) inches and \(\sigma = 2.5\) inches.

\textbf{(a)} What is the probability that a randomly selected woman is
taller than \(67\) inches?

\textbf{(b)} What height represents the \(25\)th percentile?

\textbf{(c)} What is the probability that a randomly selected woman has
a height between \(62\) and \(68\) inches?

\begin{tcolorbox}[enhanced jigsaw, toprule=.15mm, colbacktitle=quarto-callout-tip-color!10!white, colback=white, toptitle=1mm, colframe=quarto-callout-tip-color-frame, bottomrule=.15mm, title=\textcolor{quarto-callout-tip-color}{\faLightbulb}\hspace{0.5em}{Tip}, left=2mm, opacitybacktitle=0.6, bottomtitle=1mm, coltitle=black, titlerule=0mm, breakable, arc=.35mm, opacityback=0, rightrule=.15mm, leftrule=.75mm]

\textbf{Remember to standardize:} Convert to \(Z\)-scores using
\(Z = \frac{X - \mu}{\sigma}\)

For part (b), you're looking for the value \(x\) such that
\(P(X ≤ x) = 0.25\)

\end{tcolorbox}

\textbf{Work Space:}

\section{Section B: Confidence
Intervals}\label{section-b-confidence-intervals}

\emph{⏱️ Estimated time: 20 minutes}

\textbf{Problem B1: Understanding Confidence Intervals}

\textbf{(a)} Explain in your own words what a \(95\%\) confidence
interval means.

\textbf{(b)} A \(90\%\) confidence interval for the mean weight of
apples is (150g, 170g). What is the sample mean and margin of error?

\textbf{(c)} True or False: ``There is a \(95\%\) probability that the
population mean lies within our calculated \(95\%\) confidence
interval.'' Explain your reasoning.

\textbf{Work Space:}

\textbf{Problem B2: Constructing Confidence Intervals}

A sample of \(36\) students has a mean test score of \(78.5\) with a
standard deviation of \(12\).

\textbf{(a)} Construct a \(95\%\) confidence interval for the population
mean test score.

\textbf{(b)} Interpret this interval in the context of the problem.

\textbf{(c)} What would happen to the width of the interval if:

\begin{itemize}
\item
  We increased the confidence level to \(99\%\)?
\item
  We increased the sample size to \(144\)?
\end{itemize}

\begin{tcolorbox}[enhanced jigsaw, toprule=.15mm, colbacktitle=quarto-callout-tip-color!10!white, colback=white, toptitle=1mm, colframe=quarto-callout-tip-color-frame, bottomrule=.15mm, title=\textcolor{quarto-callout-tip-color}{\faLightbulb}\hspace{0.5em}{Tip}, left=2mm, opacitybacktitle=0.6, bottomtitle=1mm, coltitle=black, titlerule=0mm, breakable, arc=.35mm, opacityback=0, rightrule=.15mm, leftrule=.75mm]

\textbf{Decision Guide:}

\begin{itemize}
\item
  Use \(z\)-distribution when \(\sigma\) is \textbf{known} OR \(n ≥ 30\)
\item
  Use \(t\)-distribution when \(\sigma\) is \textbf{unknown} AND
  \(n < 30\)
\item
  For \(95\%\) CI: \(z_{0.025} = 1.96\)
\end{itemize}

\end{tcolorbox}

\textbf{Work Space:}

\section{Optional Questions}\label{optional-questions}

\textbf{Optional Problem: Conceptual Understanding}

\textbf{(a)} Explain the key difference between discrete and continuous
random variables in terms of:

\begin{itemize}
\item
  The values they can take
\item
  How we calculate probabilities
\end{itemize}

\textbf{(b)} Why do we use \(P(X = x) = 0\) for any specific value \(x\)
in a continuous distribution?

\textbf{(c)} What's the relationship between PDF and CDF for continuous
distributions?

\textbf{Work Space:}

\begin{center}\rule{0.5\linewidth}{0.5pt}\end{center}

\textbf{📋 Quick Reference:}

\textbf{Common Z-values:}

\begin{itemize}
\item
  \(90\%\) CI: \(z_{0.05} = 1.645\)
\item
  \(95\%\) CI: \(z_{0.025}\) = 1.96\$
\item
  \(99\%\) CI: \(z_{0.005}\) = 2.576\$
\end{itemize}

\textbf{Common t-values (selected):}

\begin{itemize}
\item
  \(df = 24, \alpha = 0.05: t_{0.025} = 2.064\)
\item
  \(df = 35, \alpha = 0.05: t_{0.025} = 2.030\)
\end{itemize}




\end{document}
