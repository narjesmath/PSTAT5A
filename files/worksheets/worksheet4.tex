% Options for packages loaded elsewhere
\PassOptionsToPackage{unicode}{hyperref}
\PassOptionsToPackage{hyphens}{url}
\PassOptionsToPackage{dvipsnames,svgnames,x11names}{xcolor}
%
\documentclass[
  11pt,
]{article}

\usepackage{amsmath,amssymb}
\usepackage{iftex}
\ifPDFTeX
  \usepackage[T1]{fontenc}
  \usepackage[utf8]{inputenc}
  \usepackage{textcomp} % provide euro and other symbols
\else % if luatex or xetex
  \usepackage{unicode-math}
  \defaultfontfeatures{Scale=MatchLowercase}
  \defaultfontfeatures[\rmfamily]{Ligatures=TeX,Scale=1}
\fi
\usepackage{lmodern}
\ifPDFTeX\else  
    % xetex/luatex font selection
\fi
% Use upquote if available, for straight quotes in verbatim environments
\IfFileExists{upquote.sty}{\usepackage{upquote}}{}
\IfFileExists{microtype.sty}{% use microtype if available
  \usepackage[]{microtype}
  \UseMicrotypeSet[protrusion]{basicmath} % disable protrusion for tt fonts
}{}
\makeatletter
\@ifundefined{KOMAClassName}{% if non-KOMA class
  \IfFileExists{parskip.sty}{%
    \usepackage{parskip}
  }{% else
    \setlength{\parindent}{0pt}
    \setlength{\parskip}{6pt plus 2pt minus 1pt}}
}{% if KOMA class
  \KOMAoptions{parskip=half}}
\makeatother
\usepackage{xcolor}
\usepackage[margin=1in]{geometry}
\setlength{\emergencystretch}{3em} % prevent overfull lines
\setcounter{secnumdepth}{5}
% Make \paragraph and \subparagraph free-standing
\makeatletter
\ifx\paragraph\undefined\else
  \let\oldparagraph\paragraph
  \renewcommand{\paragraph}{
    \@ifstar
      \xxxParagraphStar
      \xxxParagraphNoStar
  }
  \newcommand{\xxxParagraphStar}[1]{\oldparagraph*{#1}\mbox{}}
  \newcommand{\xxxParagraphNoStar}[1]{\oldparagraph{#1}\mbox{}}
\fi
\ifx\subparagraph\undefined\else
  \let\oldsubparagraph\subparagraph
  \renewcommand{\subparagraph}{
    \@ifstar
      \xxxSubParagraphStar
      \xxxSubParagraphNoStar
  }
  \newcommand{\xxxSubParagraphStar}[1]{\oldsubparagraph*{#1}\mbox{}}
  \newcommand{\xxxSubParagraphNoStar}[1]{\oldsubparagraph{#1}\mbox{}}
\fi
\makeatother


\providecommand{\tightlist}{%
  \setlength{\itemsep}{0pt}\setlength{\parskip}{0pt}}\usepackage{longtable,booktabs,array}
\usepackage{calc} % for calculating minipage widths
% Correct order of tables after \paragraph or \subparagraph
\usepackage{etoolbox}
\makeatletter
\patchcmd\longtable{\par}{\if@noskipsec\mbox{}\fi\par}{}{}
\makeatother
% Allow footnotes in longtable head/foot
\IfFileExists{footnotehyper.sty}{\usepackage{footnotehyper}}{\usepackage{footnote}}
\makesavenoteenv{longtable}
\usepackage{graphicx}
\makeatletter
\newsavebox\pandoc@box
\newcommand*\pandocbounded[1]{% scales image to fit in text height/width
  \sbox\pandoc@box{#1}%
  \Gscale@div\@tempa{\textheight}{\dimexpr\ht\pandoc@box+\dp\pandoc@box\relax}%
  \Gscale@div\@tempb{\linewidth}{\wd\pandoc@box}%
  \ifdim\@tempb\p@<\@tempa\p@\let\@tempa\@tempb\fi% select the smaller of both
  \ifdim\@tempa\p@<\p@\scalebox{\@tempa}{\usebox\pandoc@box}%
  \else\usebox{\pandoc@box}%
  \fi%
}
% Set default figure placement to htbp
\def\fps@figure{htbp}
\makeatother

\makeatletter
\@ifpackageloaded{tcolorbox}{}{\usepackage[skins,breakable]{tcolorbox}}
\@ifpackageloaded{fontawesome5}{}{\usepackage{fontawesome5}}
\definecolor{quarto-callout-color}{HTML}{909090}
\definecolor{quarto-callout-note-color}{HTML}{0758E5}
\definecolor{quarto-callout-important-color}{HTML}{CC1914}
\definecolor{quarto-callout-warning-color}{HTML}{EB9113}
\definecolor{quarto-callout-tip-color}{HTML}{00A047}
\definecolor{quarto-callout-caution-color}{HTML}{FC5300}
\definecolor{quarto-callout-color-frame}{HTML}{acacac}
\definecolor{quarto-callout-note-color-frame}{HTML}{4582ec}
\definecolor{quarto-callout-important-color-frame}{HTML}{d9534f}
\definecolor{quarto-callout-warning-color-frame}{HTML}{f0ad4e}
\definecolor{quarto-callout-tip-color-frame}{HTML}{02b875}
\definecolor{quarto-callout-caution-color-frame}{HTML}{fd7e14}
\makeatother
\makeatletter
\@ifpackageloaded{caption}{}{\usepackage{caption}}
\AtBeginDocument{%
\ifdefined\contentsname
  \renewcommand*\contentsname{Table of contents}
\else
  \newcommand\contentsname{Table of contents}
\fi
\ifdefined\listfigurename
  \renewcommand*\listfigurename{List of Figures}
\else
  \newcommand\listfigurename{List of Figures}
\fi
\ifdefined\listtablename
  \renewcommand*\listtablename{List of Tables}
\else
  \newcommand\listtablename{List of Tables}
\fi
\ifdefined\figurename
  \renewcommand*\figurename{Figure}
\else
  \newcommand\figurename{Figure}
\fi
\ifdefined\tablename
  \renewcommand*\tablename{Table}
\else
  \newcommand\tablename{Table}
\fi
}
\@ifpackageloaded{float}{}{\usepackage{float}}
\floatstyle{ruled}
\@ifundefined{c@chapter}{\newfloat{codelisting}{h}{lop}}{\newfloat{codelisting}{h}{lop}[chapter]}
\floatname{codelisting}{Listing}
\newcommand*\listoflistings{\listof{codelisting}{List of Listings}}
\makeatother
\makeatletter
\makeatother
\makeatletter
\@ifpackageloaded{caption}{}{\usepackage{caption}}
\@ifpackageloaded{subcaption}{}{\usepackage{subcaption}}
\makeatother

\usepackage{bookmark}

\IfFileExists{xurl.sty}{\usepackage{xurl}}{} % add URL line breaks if available
\urlstyle{same} % disable monospaced font for URLs
\hypersetup{
  pdftitle={PSTAT 5A Practice Worksheet 4},
  pdfauthor={Student Name: \_\_\_\_\_\_\_\_\_\_\_\_\_\_\_\_\_\_\_\_\_\_\_\_},
  colorlinks=true,
  linkcolor={blue},
  filecolor={Maroon},
  citecolor={Blue},
  urlcolor={Blue},
  pdfcreator={LaTeX via pandoc}}


\title{PSTAT 5A Practice Worksheet 4}
\usepackage{etoolbox}
\makeatletter
\providecommand{\subtitle}[1]{% add subtitle to \maketitle
  \apptocmd{\@title}{\par {\large #1 \par}}{}{}
}
\makeatother
\subtitle{Comprehensive Review: Discrete Random Variables and
Distributions}
\author{Student Name: \_\_\_\_\_\_\_\_\_\_\_\_\_\_\_\_\_\_\_\_\_\_\_\_}
\date{2025-07-15}

\begin{document}
\maketitle

\renewcommand*\contentsname{Table of contents}
{
\hypersetup{linkcolor=}
\setcounter{tocdepth}{3}
\tableofcontents
}

\section{Instructions and Overview}\label{instructions-and-overview}

\textbf{⏰ Time Allocation:}

\begin{itemize}
\item
  \textbf{Section A (Warm-up):} 8 minutes
\item
  \textbf{Section B (Intermediate):} 15 minutes
\item
  \textbf{Section C (Advanced):} 12 minutes
\item
  \textbf{Section D (Applications):} 15 minutes
\item
  \textbf{Total:} 50 minutes
\end{itemize}

\textbf{📝 Important Instructions:}

\begin{itemize}
\item
  Use the formulas provided for guidance
\item
  Round final answers to 4 decimal places unless otherwise specified
\item
  Identify the distribution type before calculating
\item
  Show your work for expected value and variance calculations
\item
  Use calculator as needed for factorials and combinations
\end{itemize}

\textbf{📚 Key Formulas Reference:}

\textbf{General Random Variable Properties:}

\begin{itemize}
\item
  \textbf{Expected Value:} \(E[X] = \sum_{k} k \cdot P(X = k)\)
\item
  \textbf{Variance:}
  \(\text{Var}(X) = E[X^2] - (E[X])^2 = \sum_{k} k^2 \cdot P(X = k) - \mu^2\)
\item
  \textbf{Standard Deviation:} \(\sigma = \sqrt{\text{Var}(X)}\)
\end{itemize}

\textbf{Discrete Distributions:}

\textbf{Bernoulli Distribution:} \(X \sim \text{Bernoulli}(p)\)

\begin{itemize}
\tightlist
\item
  \textbf{PMF:} \(P(X = k) = p^k(1-p)^{1-k}\) for \(k \in \{0,1\}\)
\item
  \textbf{Mean:} \(E[X] = p\)
\item
  \textbf{Variance:} \(\text{Var}(X) = p(1-p)\)
\end{itemize}

\textbf{Binomial Distribution:} \(X \sim \text{Binomial}(n,p)\)

\begin{itemize}
\tightlist
\item
  \textbf{PMF:} \(P(X = k) = \binom{n}{k} p^k (1-p)^{n-k}\) for
  \(k = 0, 1, 2, ..., n\)
\item
  \textbf{Mean:} \(E[X] = np\)
\item
  \textbf{Variance:} \(\text{Var}(X) = np(1-p)\)
\end{itemize}

\textbf{Geometric Distribution:} \(X \sim \text{Geometric}(p)\)

\begin{itemize}
\tightlist
\item
  \textbf{PMF:} \(P(X = k) = (1-p)^{k-1} p\) for \(k = 1, 2, 3, ...\)
\item
  \textbf{Mean:} \(E[X] = \frac{1}{p}\)
\item
  \textbf{Variance:} \(\text{Var}(X) = \frac{1-p}{p^2}\)
\end{itemize}

\textbf{Poisson Distribution:} \(X \sim \text{Poisson}(\lambda)\)

\begin{itemize}
\tightlist
\item
  \textbf{PMF:} \(P(X = k) = \frac{\lambda^k e^{-\lambda}}{k!}\) for
  \(k = 0, 1, 2, ...\)
\item
  \textbf{Mean:} \(E[X] = \lambda\)
\item
  \textbf{Variance:} \(\text{Var}(X) = \lambda\)
\end{itemize}

\section{Section A: Basic Concepts and
Identification}\label{section-a-basic-concepts-and-identification}

\emph{⏱️ Estimated time: 8 minutes}

\textbf{Problem A1: Distribution Identification}

For each scenario below, identify the appropriate discrete distribution
and state the parameter(s). \textbf{Do not calculate probabilities yet.}

\textbf{(a)} A fair coin is flipped until the first head appears. Let X
= number of flips needed.

\textbf{(b)} A quality control inspector tests 20 randomly selected
items from a production line where 5\% are defective. Let X = number of
defective items found.

\textbf{(c)} A website receives visitors at an average rate of 3 per
minute. Let X = number of visitors in a 2-minute period.

\textbf{(d)} A basketball player shoots one free throw with an 80\%
success rate. Let X = 1 if successful, 0 if unsuccessful.

\textbf{(e)} A student keeps taking a driving test until they pass. The
probability of passing on any attempt is 0.7. Let X = number of attempts
needed to pass.

\textbf{Work Space:}

\textbf{Problem A2: Probability Mass Function}

The random variable X has the following probability distribution:

\begin{longtable}[]{@{}llllll@{}}
\toprule\noalign{}
X & 1 & 2 & 3 & 4 & 5 \\
\midrule\noalign{}
\endhead
\bottomrule\noalign{}
\endlastfoot
P(X=k) & 0.1 & 0.3 & 0.4 & a & 0.1 \\
\end{longtable}

\textbf{(a)} Find the value of \(a\).

\textbf{(b)} Calculate \(P(X \leq 3)\).

\textbf{(c)} Calculate \(P(X > 2)\).

\textbf{Work Space:}

\section{Section B: Expected Value and
Variance}\label{section-b-expected-value-and-variance}

\emph{⏱️ Estimated time: 15 minutes}

\textbf{Problem B1: Manual Calculations}

Using the probability distribution from Problem A2, calculate:

\textbf{(a)} The expected value \(E[X]\)

\textbf{(b)} The variance \(\text{Var}(X)\)

\textbf{(c)} The standard deviation \(\sigma\)

\begin{tcolorbox}[enhanced jigsaw, coltitle=black, colback=white, titlerule=0mm, left=2mm, opacityback=0, bottomtitle=1mm, title=\textcolor{quarto-callout-tip-color}{\faLightbulb}\hspace{0.5em}{Tip}, colframe=quarto-callout-tip-color-frame, toptitle=1mm, arc=.35mm, leftrule=.75mm, toprule=.15mm, bottomrule=.15mm, breakable, rightrule=.15mm, colbacktitle=quarto-callout-tip-color!10!white, opacitybacktitle=0.6]

\textbf{Calculation Strategy:}

For expected value: \(E[X] = \sum k \cdot P(X = k)\)

For variance: First find \(E[X^2] = \sum k^2 \cdot P(X = k)\), then use
\(\text{Var}(X) = E[X^2] - (E[X])^2\)

Show your work step by step!

\end{tcolorbox}

\textbf{Work Space:}

\textbf{Problem B2: Bernoulli and Binomial Applications}

A manufacturing process produces items that are defective with
probability 0.15.

\textbf{(a)} If you select one item randomly, what is the expected value
and variance of X = number of defective items?

\textbf{(b)} If you select 25 items randomly, what is the expected
number of defective items and the standard deviation?

\begin{tcolorbox}[enhanced jigsaw, coltitle=black, colback=white, titlerule=0mm, left=2mm, opacityback=0, bottomtitle=1mm, title=\textcolor{quarto-callout-tip-color}{\faLightbulb}\hspace{0.5em}{Tip}, colframe=quarto-callout-tip-color-frame, toptitle=1mm, arc=.35mm, leftrule=.75mm, toprule=.15mm, bottomrule=.15mm, breakable, rightrule=.15mm, colbacktitle=quarto-callout-tip-color!10!white, opacitybacktitle=0.6]

Part (a) is a Bernoulli distribution. Part (b) is a Binomial
distribution. Use the formulas from the reference box!

\end{tcolorbox}

\textbf{Work Space:}

\section{Section C: Distribution
Calculations}\label{section-c-distribution-calculations}

\emph{⏱️ Estimated time: 12 minutes}

\textbf{Problem C1: Binomial Distribution}

A multiple-choice quiz has 10 questions, each with 4 choices. A student
guesses randomly on all questions.

\textbf{(a)} What is the probability the student gets exactly 3
questions correct?

\textbf{(b)} What is the probability the student gets at least 2
questions correct?

\textbf{(c)} What is the expected number of correct answers?

\begin{tcolorbox}[enhanced jigsaw, coltitle=black, colback=white, titlerule=0mm, left=2mm, opacityback=0, bottomtitle=1mm, title=\textcolor{quarto-callout-tip-color}{\faLightbulb}\hspace{0.5em}{Tip}, colframe=quarto-callout-tip-color-frame, toptitle=1mm, arc=.35mm, leftrule=.75mm, toprule=.15mm, bottomrule=.15mm, breakable, rightrule=.15mm, colbacktitle=quarto-callout-tip-color!10!white, opacitybacktitle=0.6]

This is binomial with \(n = 10\) and \(p = 0.25\) (since 1 out of 4
choices is correct).

For part (b):
\(P(X \geq 2) = 1 - P(X \leq 1) = 1 - [P(X = 0) + P(X = 1)]\)

\end{tcolorbox}

\textbf{Work Space:}

\textbf{Problem C2: Poisson Distribution}

A call center receives an average of 4 calls per minute.

\textbf{(a)} What is the probability of receiving exactly 6 calls in a
given minute?

\textbf{(b)} What is the probability of receiving no calls in a given
minute?

\textbf{(c)} What is the probability of receiving more than 2 calls in a
given minute?

\textbf{Work Space:}

\section{Section D: Advanced
Applications}\label{section-d-advanced-applications}

\emph{⏱️ Estimated time: 15 minutes}

\textbf{Problem D1: Geometric Distribution}

A software company releases updates that fix critical bugs with
probability 0.6 per update.

\textbf{(a)} What is the probability that the first successful bug fix
occurs on the 3rd update?

\textbf{(b)} What is the expected number of updates needed to get the
first successful bug fix?

\textbf{(c)} What is the probability that it takes more than 4 updates
to get the first successful bug fix?

\begin{tcolorbox}[enhanced jigsaw, coltitle=black, colback=white, titlerule=0mm, left=2mm, opacityback=0, bottomtitle=1mm, title=\textcolor{quarto-callout-tip-color}{\faLightbulb}\hspace{0.5em}{Tip}, colframe=quarto-callout-tip-color-frame, toptitle=1mm, arc=.35mm, leftrule=.75mm, toprule=.15mm, bottomrule=.15mm, breakable, rightrule=.15mm, colbacktitle=quarto-callout-tip-color!10!white, opacitybacktitle=0.6]

For part (c): \(P(X > 4) = P(X \geq 5) = (1-p)^4\) where \(p = 0.6\)

This represents the probability of 4 consecutive failures.

\end{tcolorbox}

\textbf{Work Space:}

\textbf{Problem D2: Mixed Applications}

A quality assurance team at a pharmaceutical company is testing a new
batch of medications.

\begin{itemize}
\tightlist
\item
  Each pill has a 2\% chance of being defective
\item
  They test pills one by one until they find the first defective pill
\item
  They also want to know about defects in batches of 50 pills
\end{itemize}

\textbf{(a)} What is the probability that the first defective pill is
found on the 5th test?

\textbf{(b)} What is the expected number of pills they need to test to
find the first defective one?

\textbf{(c)} In a batch of 50 pills, what is the probability that
exactly 2 pills are defective?

\textbf{(d)} In a batch of 50 pills, what is the expected number of
defective pills?

\begin{tcolorbox}[enhanced jigsaw, coltitle=black, colback=white, titlerule=0mm, left=2mm, opacityback=0, bottomtitle=1mm, title=\textcolor{quarto-callout-tip-color}{\faLightbulb}\hspace{0.5em}{Tip}, colframe=quarto-callout-tip-color-frame, toptitle=1mm, arc=.35mm, leftrule=.75mm, toprule=.15mm, bottomrule=.15mm, breakable, rightrule=.15mm, colbacktitle=quarto-callout-tip-color!10!white, opacitybacktitle=0.6]

Parts (a) and (b) use geometric distribution. Parts (c) and (d) use
binomial distribution.

Be careful with the parameter \(p = 0.02\) for both distributions.

\end{tcolorbox}

\textbf{Work Space:}

\section{Reflection Questions}\label{reflection-questions}

\textbf{Problem E: Conceptual Understanding}

\textbf{(a)} Explain the key difference between a Binomial distribution
and a Geometric distribution in terms of what they count.

\textbf{(b)} When would you use a Poisson distribution instead of a
Binomial distribution?

\textbf{(c)} If \(X \sim \text{Binomial}(n, p)\), under what conditions
would the variance be maximized?

\textbf{Work Space:}




\end{document}
