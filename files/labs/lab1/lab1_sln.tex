% Options for packages loaded elsewhere
% Options for packages loaded elsewhere
\PassOptionsToPackage{unicode}{hyperref}
\PassOptionsToPackage{hyphens}{url}
\PassOptionsToPackage{dvipsnames,svgnames,x11names}{xcolor}
%
\documentclass[
  11pt,
]{article}
\usepackage{xcolor}
\usepackage[margin=1in]{geometry}
\usepackage{amsmath,amssymb}
\setcounter{secnumdepth}{5}
\usepackage{iftex}
\ifPDFTeX
  \usepackage[T1]{fontenc}
  \usepackage[utf8]{inputenc}
  \usepackage{textcomp} % provide euro and other symbols
\else % if luatex or xetex
  \usepackage{unicode-math} % this also loads fontspec
  \defaultfontfeatures{Scale=MatchLowercase}
  \defaultfontfeatures[\rmfamily]{Ligatures=TeX,Scale=1}
\fi
\usepackage{lmodern}
\ifPDFTeX\else
  % xetex/luatex font selection
\fi
% Use upquote if available, for straight quotes in verbatim environments
\IfFileExists{upquote.sty}{\usepackage{upquote}}{}
\IfFileExists{microtype.sty}{% use microtype if available
  \usepackage[]{microtype}
  \UseMicrotypeSet[protrusion]{basicmath} % disable protrusion for tt fonts
}{}
\makeatletter
\@ifundefined{KOMAClassName}{% if non-KOMA class
  \IfFileExists{parskip.sty}{%
    \usepackage{parskip}
  }{% else
    \setlength{\parindent}{0pt}
    \setlength{\parskip}{6pt plus 2pt minus 1pt}}
}{% if KOMA class
  \KOMAoptions{parskip=half}}
\makeatother
% Make \paragraph and \subparagraph free-standing
\makeatletter
\ifx\paragraph\undefined\else
  \let\oldparagraph\paragraph
  \renewcommand{\paragraph}{
    \@ifstar
      \xxxParagraphStar
      \xxxParagraphNoStar
  }
  \newcommand{\xxxParagraphStar}[1]{\oldparagraph*{#1}\mbox{}}
  \newcommand{\xxxParagraphNoStar}[1]{\oldparagraph{#1}\mbox{}}
\fi
\ifx\subparagraph\undefined\else
  \let\oldsubparagraph\subparagraph
  \renewcommand{\subparagraph}{
    \@ifstar
      \xxxSubParagraphStar
      \xxxSubParagraphNoStar
  }
  \newcommand{\xxxSubParagraphStar}[1]{\oldsubparagraph*{#1}\mbox{}}
  \newcommand{\xxxSubParagraphNoStar}[1]{\oldsubparagraph{#1}\mbox{}}
\fi
\makeatother

\usepackage{color}
\usepackage{fancyvrb}
\newcommand{\VerbBar}{|}
\newcommand{\VERB}{\Verb[commandchars=\\\{\}]}
\DefineVerbatimEnvironment{Highlighting}{Verbatim}{commandchars=\\\{\}}
% Add ',fontsize=\small' for more characters per line
\usepackage{framed}
\definecolor{shadecolor}{RGB}{241,243,245}
\newenvironment{Shaded}{\begin{snugshade}}{\end{snugshade}}
\newcommand{\AlertTok}[1]{\textcolor[rgb]{0.68,0.00,0.00}{#1}}
\newcommand{\AnnotationTok}[1]{\textcolor[rgb]{0.37,0.37,0.37}{#1}}
\newcommand{\AttributeTok}[1]{\textcolor[rgb]{0.40,0.45,0.13}{#1}}
\newcommand{\BaseNTok}[1]{\textcolor[rgb]{0.68,0.00,0.00}{#1}}
\newcommand{\BuiltInTok}[1]{\textcolor[rgb]{0.00,0.23,0.31}{#1}}
\newcommand{\CharTok}[1]{\textcolor[rgb]{0.13,0.47,0.30}{#1}}
\newcommand{\CommentTok}[1]{\textcolor[rgb]{0.37,0.37,0.37}{#1}}
\newcommand{\CommentVarTok}[1]{\textcolor[rgb]{0.37,0.37,0.37}{\textit{#1}}}
\newcommand{\ConstantTok}[1]{\textcolor[rgb]{0.56,0.35,0.01}{#1}}
\newcommand{\ControlFlowTok}[1]{\textcolor[rgb]{0.00,0.23,0.31}{\textbf{#1}}}
\newcommand{\DataTypeTok}[1]{\textcolor[rgb]{0.68,0.00,0.00}{#1}}
\newcommand{\DecValTok}[1]{\textcolor[rgb]{0.68,0.00,0.00}{#1}}
\newcommand{\DocumentationTok}[1]{\textcolor[rgb]{0.37,0.37,0.37}{\textit{#1}}}
\newcommand{\ErrorTok}[1]{\textcolor[rgb]{0.68,0.00,0.00}{#1}}
\newcommand{\ExtensionTok}[1]{\textcolor[rgb]{0.00,0.23,0.31}{#1}}
\newcommand{\FloatTok}[1]{\textcolor[rgb]{0.68,0.00,0.00}{#1}}
\newcommand{\FunctionTok}[1]{\textcolor[rgb]{0.28,0.35,0.67}{#1}}
\newcommand{\ImportTok}[1]{\textcolor[rgb]{0.00,0.46,0.62}{#1}}
\newcommand{\InformationTok}[1]{\textcolor[rgb]{0.37,0.37,0.37}{#1}}
\newcommand{\KeywordTok}[1]{\textcolor[rgb]{0.00,0.23,0.31}{\textbf{#1}}}
\newcommand{\NormalTok}[1]{\textcolor[rgb]{0.00,0.23,0.31}{#1}}
\newcommand{\OperatorTok}[1]{\textcolor[rgb]{0.37,0.37,0.37}{#1}}
\newcommand{\OtherTok}[1]{\textcolor[rgb]{0.00,0.23,0.31}{#1}}
\newcommand{\PreprocessorTok}[1]{\textcolor[rgb]{0.68,0.00,0.00}{#1}}
\newcommand{\RegionMarkerTok}[1]{\textcolor[rgb]{0.00,0.23,0.31}{#1}}
\newcommand{\SpecialCharTok}[1]{\textcolor[rgb]{0.37,0.37,0.37}{#1}}
\newcommand{\SpecialStringTok}[1]{\textcolor[rgb]{0.13,0.47,0.30}{#1}}
\newcommand{\StringTok}[1]{\textcolor[rgb]{0.13,0.47,0.30}{#1}}
\newcommand{\VariableTok}[1]{\textcolor[rgb]{0.07,0.07,0.07}{#1}}
\newcommand{\VerbatimStringTok}[1]{\textcolor[rgb]{0.13,0.47,0.30}{#1}}
\newcommand{\WarningTok}[1]{\textcolor[rgb]{0.37,0.37,0.37}{\textit{#1}}}

\usepackage{longtable,booktabs,array}
\usepackage{calc} % for calculating minipage widths
% Correct order of tables after \paragraph or \subparagraph
\usepackage{etoolbox}
\makeatletter
\patchcmd\longtable{\par}{\if@noskipsec\mbox{}\fi\par}{}{}
\makeatother
% Allow footnotes in longtable head/foot
\IfFileExists{footnotehyper.sty}{\usepackage{footnotehyper}}{\usepackage{footnote}}
\makesavenoteenv{longtable}
\usepackage{graphicx}
\makeatletter
\newsavebox\pandoc@box
\newcommand*\pandocbounded[1]{% scales image to fit in text height/width
  \sbox\pandoc@box{#1}%
  \Gscale@div\@tempa{\textheight}{\dimexpr\ht\pandoc@box+\dp\pandoc@box\relax}%
  \Gscale@div\@tempb{\linewidth}{\wd\pandoc@box}%
  \ifdim\@tempb\p@<\@tempa\p@\let\@tempa\@tempb\fi% select the smaller of both
  \ifdim\@tempa\p@<\p@\scalebox{\@tempa}{\usebox\pandoc@box}%
  \else\usebox{\pandoc@box}%
  \fi%
}
% Set default figure placement to htbp
\def\fps@figure{htbp}
\makeatother





\setlength{\emergencystretch}{3em} % prevent overfull lines

\providecommand{\tightlist}{%
  \setlength{\itemsep}{0pt}\setlength{\parskip}{0pt}}



 


\makeatletter
\@ifpackageloaded{caption}{}{\usepackage{caption}}
\AtBeginDocument{%
\ifdefined\contentsname
  \renewcommand*\contentsname{Table of contents}
\else
  \newcommand\contentsname{Table of contents}
\fi
\ifdefined\listfigurename
  \renewcommand*\listfigurename{List of Figures}
\else
  \newcommand\listfigurename{List of Figures}
\fi
\ifdefined\listtablename
  \renewcommand*\listtablename{List of Tables}
\else
  \newcommand\listtablename{List of Tables}
\fi
\ifdefined\figurename
  \renewcommand*\figurename{Figure}
\else
  \newcommand\figurename{Figure}
\fi
\ifdefined\tablename
  \renewcommand*\tablename{Table}
\else
  \newcommand\tablename{Table}
\fi
}
\@ifpackageloaded{float}{}{\usepackage{float}}
\floatstyle{ruled}
\@ifundefined{c@chapter}{\newfloat{codelisting}{h}{lop}}{\newfloat{codelisting}{h}{lop}[chapter]}
\floatname{codelisting}{Listing}
\newcommand*\listoflistings{\listof{codelisting}{List of Listings}}
\makeatother
\makeatletter
\makeatother
\makeatletter
\@ifpackageloaded{caption}{}{\usepackage{caption}}
\@ifpackageloaded{subcaption}{}{\usepackage{subcaption}}
\makeatother
\makeatletter
\@ifpackageloaded{sidenotes}{}{\usepackage{sidenotes}}
\@ifpackageloaded{marginnote}{}{\usepackage{marginnote}}
\makeatother
\usepackage{bookmark}
\IfFileExists{xurl.sty}{\usepackage{xurl}}{} % add URL line breaks if available
\urlstyle{same}
\hypersetup{
  pdftitle={PSTAT 5A Lab 1 - SOLUTIONS},
  pdfauthor={Solution Key},
  colorlinks=true,
  linkcolor={blue},
  filecolor={Maroon},
  citecolor={Blue},
  urlcolor={Blue},
  pdfcreator={LaTeX via pandoc}}


\title{PSTAT 5A Lab 1 - SOLUTIONS}
\usepackage{etoolbox}
\makeatletter
\providecommand{\subtitle}[1]{% add subtitle to \maketitle
  \apptocmd{\@title}{\par {\large #1 \par}}{}{}
}
\makeatother
\subtitle{Introduction to Python and JupyterHub}
\author{Solution Key}
\date{2025-07-23}
\begin{document}
\maketitle

\renewcommand*\contentsname{Table of contents}
{
\hypersetup{linkcolor=}
\setcounter{tocdepth}{3}
\tableofcontents
}

\marginnote{\begin{footnotesize}

\includegraphics[width=1.5625in,height=\textheight,keepaspectratio]{../../../img/logo.png}

\end{footnotesize}}

\section{Lab 1 Solutions}\label{lab-1-solutions}

This document provides complete solutions to all tasks in PSTAT 5A Lab
1.

\begin{center}\rule{0.5\linewidth}{0.5pt}\end{center}

\subsection{Task 1 Solution}\label{task-1-solution}

\textbf{Objective:} Rename your notebook from \texttt{Untitled} to
\texttt{Lab1}

\textbf{Steps:}

\begin{itemize}
\item
  Located the notebook in the file browser (appears as
  \texttt{Untitled.ipynb})
\item
  Right-clicked on the notebook name
\item
  Selected ``Rename'' from the context menu
\item
  Changed the name to \texttt{Lab1}
\item
  Pressed Enter to confirm
\item
  Verified the title bar now shows \texttt{Lab1.ipynb}
\end{itemize}

\textbf{Expected Result:}

Your notebook should now be named \texttt{Lab1.ipynb} and this should be
visible in both the file browser and the title bar.

\begin{center}\rule{0.5\linewidth}{0.5pt}\end{center}

\subsection{Task 2 Solution}\label{task-2-solution}

\textbf{Objective:} Create a Markdown cell with heading and a Code cell
with basic arithmetic

\textbf{Markdown Cell:}

\begin{Shaded}
\begin{Highlighting}[]
\NormalTok{\# Task 2}
\end{Highlighting}
\end{Shaded}

\textbf{Code Cell:}

\begin{Shaded}
\begin{Highlighting}[]
\DecValTok{2} \OperatorTok{+} \DecValTok{2}
\end{Highlighting}
\end{Shaded}

\begin{verbatim}
4
\end{verbatim}

\textbf{Expected Result:}

\begin{itemize}
\item
  The code cell executes the arithmetic operation
\item
  Python displays the result \texttt{4} below the cell
\item
  A new empty code cell automatically appears below
\item
  The cell is marked as executed with a number like \texttt{{[}1{]}}
\end{itemize}

\begin{center}\rule{0.5\linewidth}{0.5pt}\end{center}

\subsection{Task 3 Solution}\label{task-3-solution}

\textbf{Objective:} Demonstrate syntax error and understand error
messages

\textbf{Markdown Cell:}

\begin{Shaded}
\begin{Highlighting}[]
\NormalTok{\# Task 3}
\end{Highlighting}
\end{Shaded}

\textbf{Code Cell with Intentional Error:}

\begin{Shaded}
\begin{Highlighting}[]
\DecValTok{2}\NormalTok{ plus }\DecValTok{2}
\end{Highlighting}
\end{Shaded}

\textbf{Expected Error Output:}

\begin{verbatim}
  Cell In[2], line 1
    2 plus 2
      ^^^^
SyntaxError: invalid syntax
\end{verbatim}

\textbf{Explanation:}

\begin{itemize}
\item
  Python doesn't understand \texttt{plus} as an operator
\item
  The \texttt{\^{}\^{}\^{}\^{}} points to where Python detected the
  problem
\item
  The error message tells us it's a \texttt{SyntaxError} meaning invalid
  Python syntax
\item
  In Python, we must use \texttt{+} for addition, not the word
  \texttt{plus}
\end{itemize}

\textbf{Corrected Version:}

\begin{Shaded}
\begin{Highlighting}[]
\DecValTok{2} \OperatorTok{+} \DecValTok{2}  \CommentTok{\# This works correctly}
\end{Highlighting}
\end{Shaded}

\begin{verbatim}
4
\end{verbatim}

\begin{center}\rule{0.5\linewidth}{0.5pt}\end{center}

\subsection{Task 4 Solution}\label{task-4-solution}

\textbf{Objective:} Compute mathematical expressions using Python**

\textbf{Problem 1:} \(\frac{2 + 3}{4 + 5^6}\)

\textbf{Python Code:}

\begin{Shaded}
\begin{Highlighting}[]
\NormalTok{(}\DecValTok{2} \OperatorTok{+} \DecValTok{3}\NormalTok{) }\OperatorTok{/}\NormalTok{ (}\DecValTok{4} \OperatorTok{+} \DecValTok{5}\OperatorTok{**}\DecValTok{6}\NormalTok{)}
\end{Highlighting}
\end{Shaded}

\begin{verbatim}
0.0003199181009661527
\end{verbatim}

\textbf{Step by Step:}

\begin{Shaded}
\begin{Highlighting}[]
\NormalTok{numerator }\OperatorTok{=} \DecValTok{2} \OperatorTok{+} \DecValTok{3}
\BuiltInTok{print}\NormalTok{(}\SpecialStringTok{f"Numerator: }\SpecialCharTok{\{}\NormalTok{numerator}\SpecialCharTok{\}}\SpecialStringTok{"}\NormalTok{)}

\NormalTok{denominator }\OperatorTok{=} \DecValTok{4} \OperatorTok{+} \DecValTok{5}\OperatorTok{**}\DecValTok{6}
\BuiltInTok{print}\NormalTok{(}\SpecialStringTok{f"Denominator: }\SpecialCharTok{\{}\NormalTok{denominator}\SpecialCharTok{\}}\SpecialStringTok{"}\NormalTok{)}

\NormalTok{result }\OperatorTok{=}\NormalTok{ numerator }\OperatorTok{/}\NormalTok{ denominator}
\BuiltInTok{print}\NormalTok{(}\SpecialStringTok{f"Final result: }\SpecialCharTok{\{}\NormalTok{result}\SpecialCharTok{\}}\SpecialStringTok{"}\NormalTok{)}
\end{Highlighting}
\end{Shaded}

\begin{verbatim}
Numerator: 5
Denominator: 15629
Final result: 0.0003199181009661527
\end{verbatim}

\textbf{Problem 2:} \((1 - 3 \cdot 4^5)^6\)

\textbf{Python Code:}

\begin{Shaded}
\begin{Highlighting}[]
\NormalTok{(}\DecValTok{1} \OperatorTok{{-}} \DecValTok{3} \OperatorTok{*} \DecValTok{4}\OperatorTok{**}\DecValTok{5}\NormalTok{)}\OperatorTok{**}\DecValTok{6}
\end{Highlighting}
\end{Shaded}

\begin{verbatim}
838839550121163601921
\end{verbatim}

\textbf{Step by Step:}

\begin{Shaded}
\begin{Highlighting}[]
\NormalTok{inner\_exponent }\OperatorTok{=} \DecValTok{4}\OperatorTok{**}\DecValTok{5}
\BuiltInTok{print}\NormalTok{(}\SpecialStringTok{f"4\^{}5 = }\SpecialCharTok{\{}\NormalTok{inner\_exponent}\SpecialCharTok{\}}\SpecialStringTok{"}\NormalTok{)}

\NormalTok{multiplication }\OperatorTok{=} \DecValTok{3} \OperatorTok{*}\NormalTok{ inner\_exponent}
\BuiltInTok{print}\NormalTok{(}\SpecialStringTok{f"3 * 4\^{}5 = }\SpecialCharTok{\{}\NormalTok{multiplication}\SpecialCharTok{\}}\SpecialStringTok{"}\NormalTok{)}

\NormalTok{subtraction }\OperatorTok{=} \DecValTok{1} \OperatorTok{{-}}\NormalTok{ multiplication}
\BuiltInTok{print}\NormalTok{(}\SpecialStringTok{f"1 {-} 3 * 4\^{}5 = }\SpecialCharTok{\{}\NormalTok{subtraction}\SpecialCharTok{\}}\SpecialStringTok{"}\NormalTok{)}

\NormalTok{final\_result }\OperatorTok{=}\NormalTok{ subtraction}\OperatorTok{**}\DecValTok{6}
\BuiltInTok{print}\NormalTok{(}\SpecialStringTok{f"(1 {-} 3 * 4\^{}5)\^{}6 = }\SpecialCharTok{\{}\NormalTok{final\_result}\SpecialCharTok{\}}\SpecialStringTok{"}\NormalTok{)}
\end{Highlighting}
\end{Shaded}

\begin{verbatim}
4^5 = 1024
3 * 4^5 = 3072
1 - 3 * 4^5 = -3071
(1 - 3 * 4^5)^6 = 838839550121163601921
\end{verbatim}

\begin{center}\rule{0.5\linewidth}{0.5pt}\end{center}

\subsection{Task 5 Solution}\label{task-5-solution}

\textbf{Objective:} Understand module importing and fix NameError

\textbf{Step 1: Code that produces error}

\begin{Shaded}
\begin{Highlighting}[]
\NormalTok{sin(}\DecValTok{1}\NormalTok{)}
\end{Highlighting}
\end{Shaded}

\textbf{Expected Error:}

\begin{verbatim}
NameError: name 'sin' is not defined
\end{verbatim}

\textbf{Explanation:} Python doesn't recognize \texttt{sin} because the
math functions aren't loaded by default.

\textbf{Step 2: Import module and retry}

\begin{Shaded}
\begin{Highlighting}[]
\ImportTok{from}\NormalTok{ math }\ImportTok{import} \OperatorTok{*}
\NormalTok{sin(}\DecValTok{1}\NormalTok{)}
\end{Highlighting}
\end{Shaded}

\begin{verbatim}
0.8414709848078965
\end{verbatim}

\textbf{Alternative Solutions:}

\begin{Shaded}
\begin{Highlighting}[]
\CommentTok{\# Method 1: Import specific function}
\ImportTok{from}\NormalTok{ math }\ImportTok{import}\NormalTok{ sin}
\BuiltInTok{print}\NormalTok{(sin(}\DecValTok{1}\NormalTok{))}

\CommentTok{\# Method 2: Import entire module}
\ImportTok{import}\NormalTok{ math}
\BuiltInTok{print}\NormalTok{(math.sin(}\DecValTok{1}\NormalTok{))}

\CommentTok{\# Method 3: Import with alias}
\ImportTok{import}\NormalTok{ math }\ImportTok{as}\NormalTok{ m}
\BuiltInTok{print}\NormalTok{(m.sin(}\DecValTok{1}\NormalTok{))}
\end{Highlighting}
\end{Shaded}

\begin{verbatim}
0.8414709848078965
0.8414709848078965
0.8414709848078965
\end{verbatim}

All produce the same result: \texttt{0.8414709848078965}

\begin{center}\rule{0.5\linewidth}{0.5pt}\end{center}

\subsection{Task 6 Solution}\label{task-6-solution}

\textbf{Objective:} Understand Python case sensitivity

\textbf{Step 1: Variable assignment}

\begin{Shaded}
\begin{Highlighting}[]
\NormalTok{my\_variable }\OperatorTok{=} \DecValTok{5}
\end{Highlighting}
\end{Shaded}

\textbf{Step 2: Wrong capitalization}

\begin{Shaded}
\begin{Highlighting}[]
\BuiltInTok{print}\NormalTok{(My\_variable)}
\end{Highlighting}
\end{Shaded}

\textbf{Expected Error:}

\begin{verbatim}
NameError: name 'My_variable' is not defined
\end{verbatim}

\textbf{Explanation:} Python is case-sensitive, so \texttt{My\_variable}
≠ \texttt{my\_variable}

\textbf{Step 3: Correct capitalization}

\begin{Shaded}
\begin{Highlighting}[]
\BuiltInTok{print}\NormalTok{(my\_variable)}
\end{Highlighting}
\end{Shaded}

\begin{verbatim}
5
\end{verbatim}

\textbf{Additional Examples:}

\begin{Shaded}
\begin{Highlighting}[]
\CommentTok{\# These are all different variables in Python}
\NormalTok{my\_variable }\OperatorTok{=} \DecValTok{5}
\NormalTok{My\_variable }\OperatorTok{=} \DecValTok{10}
\NormalTok{MY\_VARIABLE }\OperatorTok{=} \DecValTok{15}
\NormalTok{my\_Variable }\OperatorTok{=} \DecValTok{20}

\BuiltInTok{print}\NormalTok{(}\SpecialStringTok{f"my\_variable = }\SpecialCharTok{\{}\NormalTok{my\_variable}\SpecialCharTok{\}}\SpecialStringTok{"}\NormalTok{)}
\BuiltInTok{print}\NormalTok{(}\SpecialStringTok{f"My\_variable = }\SpecialCharTok{\{}\NormalTok{My\_variable}\SpecialCharTok{\}}\SpecialStringTok{"}\NormalTok{)}
\BuiltInTok{print}\NormalTok{(}\SpecialStringTok{f"MY\_VARIABLE = }\SpecialCharTok{\{}\NormalTok{MY\_VARIABLE}\SpecialCharTok{\}}\SpecialStringTok{"}\NormalTok{)}
\BuiltInTok{print}\NormalTok{(}\SpecialStringTok{f"my\_Variable = }\SpecialCharTok{\{}\NormalTok{my\_Variable}\SpecialCharTok{\}}\SpecialStringTok{"}\NormalTok{)}
\end{Highlighting}
\end{Shaded}

\begin{verbatim}
my_variable = 5
My_variable = 10
MY_VARIABLE = 15
my_Variable = 20
\end{verbatim}

\begin{center}\rule{0.5\linewidth}{0.5pt}\end{center}

\subsection{Task 7 Solution}\label{task-7-solution}

\textbf{Objective:} Add descriptive comments to previous code

\textbf{Examples of well-commented code from previous tasks:}

\begin{Shaded}
\begin{Highlighting}[]
\CommentTok{\# Task 2: Basic arithmetic}
\DecValTok{2} \OperatorTok{+} \DecValTok{2}  \CommentTok{\# Adding two integers}

\CommentTok{\# Task 4: Complex mathematical expression}
\CommentTok{\# Calculate (2 + 3) / (4 + 5\^{}6)}
\NormalTok{numerator }\OperatorTok{=} \DecValTok{2} \OperatorTok{+} \DecValTok{3}  \CommentTok{\# Sum of 2 and 3}
\NormalTok{denominator }\OperatorTok{=} \DecValTok{4} \OperatorTok{+} \DecValTok{5}\OperatorTok{**}\DecValTok{6}  \CommentTok{\# 4 plus 5 to the 6th power}
\NormalTok{result }\OperatorTok{=}\NormalTok{ numerator }\OperatorTok{/}\NormalTok{ denominator  }\CommentTok{\# Final division}
\BuiltInTok{print}\NormalTok{(}\SpecialStringTok{f"Result: }\SpecialCharTok{\{}\NormalTok{result}\SpecialCharTok{\}}\SpecialStringTok{"}\NormalTok{)}

\CommentTok{\# Task 5: Import math module and use sin function}
\ImportTok{from}\NormalTok{ math }\ImportTok{import} \OperatorTok{*}  \CommentTok{\# Import all math functions}
\NormalTok{angle\_in\_radians }\OperatorTok{=} \DecValTok{1}  \CommentTok{\# Input angle in radians}
\NormalTok{sine\_value }\OperatorTok{=}\NormalTok{ sin(angle\_in\_radians)  }\CommentTok{\# Calculate sine}
\BuiltInTok{print}\NormalTok{(}\SpecialStringTok{f"sin(1) = }\SpecialCharTok{\{}\NormalTok{sine\_value}\SpecialCharTok{\}}\SpecialStringTok{"}\NormalTok{)}

\CommentTok{\# Task 6: Variable assignment with proper naming}
\NormalTok{my\_variable }\OperatorTok{=} \DecValTok{5}  \CommentTok{\# Store the value 5 in my\_variable}
\BuiltInTok{print}\NormalTok{(my\_variable)  }\CommentTok{\# Display the value}

\CommentTok{"""}
\CommentTok{This is a multi{-}line comment.}
\CommentTok{It can span multiple lines and is useful}
\CommentTok{for longer explanations or documentation.}
\CommentTok{"""}
\end{Highlighting}
\end{Shaded}

\textbf{Good commenting practices demonstrated:}

\begin{itemize}
\item
  Explain what the code does
\item
  Clarify complex calculations
\item
  Document variable purposes
\item
  Use both inline (\texttt{\#}) and block (\texttt{"""}) comments
\end{itemize}

\begin{center}\rule{0.5\linewidth}{0.5pt}\end{center}

\subsection{Task 8 Solution}\label{task-8-solution}

\textbf{Objective:} Explore Python data types using the \texttt{type()}
function

\textbf{Code and Expected Outputs:}

\begin{Shaded}
\begin{Highlighting}[]
\BuiltInTok{print}\NormalTok{(}\BuiltInTok{type}\NormalTok{(}\DecValTok{1}\NormalTok{))           }\CommentTok{\# Output: \textless{}class \textquotesingle{}int\textquotesingle{}\textgreater{}}
\BuiltInTok{print}\NormalTok{(}\BuiltInTok{type}\NormalTok{(}\FloatTok{1.1}\NormalTok{))         }\CommentTok{\# Output: \textless{}class \textquotesingle{}float\textquotesingle{}\textgreater{}}
\BuiltInTok{print}\NormalTok{(}\BuiltInTok{type}\NormalTok{(}\StringTok{"hello"}\NormalTok{))     }\CommentTok{\# Output: \textless{}class \textquotesingle{}str\textquotesingle{}\textgreater{}}
\end{Highlighting}
\end{Shaded}

\begin{verbatim}
<class 'int'>
<class 'float'>
<class 'str'>
\end{verbatim}

\textbf{Additional Examples:}

\begin{Shaded}
\begin{Highlighting}[]
\CommentTok{\# More data type examples}
\BuiltInTok{print}\NormalTok{(}\StringTok{"Integer:"}\NormalTok{, }\BuiltInTok{type}\NormalTok{(}\DecValTok{42}\NormalTok{))}
\BuiltInTok{print}\NormalTok{(}\StringTok{"Float:"}\NormalTok{, }\BuiltInTok{type}\NormalTok{(}\FloatTok{3.14159}\NormalTok{))}
\BuiltInTok{print}\NormalTok{(}\StringTok{"String with single quotes:"}\NormalTok{, }\BuiltInTok{type}\NormalTok{(}\StringTok{\textquotesingle{}Python\textquotesingle{}}\NormalTok{))}
\BuiltInTok{print}\NormalTok{(}\StringTok{"String with double quotes:"}\NormalTok{, }\BuiltInTok{type}\NormalTok{(}\StringTok{"Programming"}\NormalTok{))}
\BuiltInTok{print}\NormalTok{(}\StringTok{"Boolean True:"}\NormalTok{, }\BuiltInTok{type}\NormalTok{(}\VariableTok{True}\NormalTok{))}
\BuiltInTok{print}\NormalTok{(}\StringTok{"Boolean False:"}\NormalTok{, }\BuiltInTok{type}\NormalTok{(}\VariableTok{False}\NormalTok{))}
\BuiltInTok{print}\NormalTok{(}\StringTok{"List:"}\NormalTok{, }\BuiltInTok{type}\NormalTok{([}\DecValTok{1}\NormalTok{, }\DecValTok{2}\NormalTok{, }\DecValTok{3}\NormalTok{]))}
\BuiltInTok{print}\NormalTok{(}\StringTok{"Tuple:"}\NormalTok{, }\BuiltInTok{type}\NormalTok{((}\DecValTok{1}\NormalTok{, }\DecValTok{2}\NormalTok{, }\DecValTok{3}\NormalTok{)))}
\BuiltInTok{print}\NormalTok{(}\StringTok{"Dictionary:"}\NormalTok{, }\BuiltInTok{type}\NormalTok{(\{}\StringTok{"key"}\NormalTok{: }\StringTok{"value"}\NormalTok{\}))}
\end{Highlighting}
\end{Shaded}

\begin{verbatim}
Integer: <class 'int'>
Float: <class 'float'>
String with single quotes: <class 'str'>
String with double quotes: <class 'str'>
Boolean True: <class 'bool'>
Boolean False: <class 'bool'>
List: <class 'list'>
Tuple: <class 'tuple'>
Dictionary: <class 'dict'>
\end{verbatim}

\begin{center}\rule{0.5\linewidth}{0.5pt}\end{center}

\subsection{Task 9 Solution}\label{task-9-solution}

\textbf{Objective:} Practice variable assignment, updating, and
calculations

\textbf{Markdown cell:}

\begin{Shaded}
\begin{Highlighting}[]
\NormalTok{\# Task 9}
\end{Highlighting}
\end{Shaded}

\textbf{Step 2: Initial variable assignments}

\begin{Shaded}
\begin{Highlighting}[]
\NormalTok{course }\OperatorTok{=} \StringTok{"PSTAT 5A"}
\NormalTok{num\_sections }\OperatorTok{=} \DecValTok{4}
\NormalTok{section\_capacity }\OperatorTok{=} \DecValTok{25}
\end{Highlighting}
\end{Shaded}

\textbf{Step 3: Update num\_sections (correct approach)}

\begin{Shaded}
\begin{Highlighting}[]
\NormalTok{num\_sections }\OperatorTok{=}\NormalTok{ num\_sections }\OperatorTok{+} \DecValTok{1}
\BuiltInTok{print}\NormalTok{(}\SpecialStringTok{f"Updated number of sections: }\SpecialCharTok{\{}\NormalTok{num\_sections}\SpecialCharTok{\}}\SpecialStringTok{"}\NormalTok{)}
\CommentTok{\# Alternative: num\_sections += 1}
\CommentTok{\# Alternative: num\_sections = 4 + 1}
\end{Highlighting}
\end{Shaded}

\begin{verbatim}
Updated number of sections: 5
\end{verbatim}

\textbf{Step 4: Predict and test expressions}

\begin{Shaded}
\begin{Highlighting}[]
\BuiltInTok{print}\NormalTok{(}\BuiltInTok{type}\NormalTok{(course))           }\CommentTok{\# Expected: \textless{}class \textquotesingle{}str\textquotesingle{}\textgreater{}}
\BuiltInTok{print}\NormalTok{(}\BuiltInTok{type}\NormalTok{(num\_sections))     }\CommentTok{\# Expected: \textless{}class \textquotesingle{}int\textquotesingle{}\textgreater{}}
\BuiltInTok{print}\NormalTok{(num\_sections }\OperatorTok{*}\NormalTok{ section\_capacity) }\CommentTok{\# Expected: 125}
\end{Highlighting}
\end{Shaded}

\begin{verbatim}
<class 'str'>
<class 'int'>
125
\end{verbatim}

\textbf{Step 5: Calculate course capacity}

\begin{Shaded}
\begin{Highlighting}[]
\NormalTok{course\_capacity }\OperatorTok{=}\NormalTok{ num\_sections }\OperatorTok{*}\NormalTok{ section\_capacity}
\BuiltInTok{print}\NormalTok{(}\SpecialStringTok{f"Course: }\SpecialCharTok{\{}\NormalTok{course}\SpecialCharTok{\}}\SpecialStringTok{"}\NormalTok{)}
\BuiltInTok{print}\NormalTok{(}\SpecialStringTok{f"Number of sections: }\SpecialCharTok{\{}\NormalTok{num\_sections}\SpecialCharTok{\}}\SpecialStringTok{"}\NormalTok{)}
\BuiltInTok{print}\NormalTok{(}\SpecialStringTok{f"Capacity per section: }\SpecialCharTok{\{}\NormalTok{section\_capacity}\SpecialCharTok{\}}\SpecialStringTok{"}\NormalTok{)}
\BuiltInTok{print}\NormalTok{(}\SpecialStringTok{f"Total course capacity: }\SpecialCharTok{\{}\NormalTok{course\_capacity}\SpecialCharTok{\}}\SpecialStringTok{"}\NormalTok{)}
\end{Highlighting}
\end{Shaded}

\begin{verbatim}
Course: PSTAT 5A
Number of sections: 5
Capacity per section: 25
Total course capacity: 125
\end{verbatim}

\textbf{Complete Solution with Comments:}

\begin{Shaded}
\begin{Highlighting}[]
\CommentTok{\# Step 2: Initial variable assignments}
\NormalTok{course }\OperatorTok{=} \StringTok{"PSTAT 5A"}          \CommentTok{\# Course name as string}
\NormalTok{num\_sections }\OperatorTok{=} \DecValTok{4}             \CommentTok{\# Initial number of sections}
\NormalTok{section\_capacity }\OperatorTok{=} \DecValTok{25}        \CommentTok{\# Maximum students per section}

\CommentTok{\# Step 3: A new section has been added}
\NormalTok{num\_sections }\OperatorTok{=}\NormalTok{ num\_sections }\OperatorTok{+} \DecValTok{1}  \CommentTok{\# Increment by 1, now equals 5}

\CommentTok{\# Step 4: Testing expressions with predictions}
\BuiltInTok{print}\NormalTok{(}\StringTok{"Testing type() function:"}\NormalTok{)}
\BuiltInTok{print}\NormalTok{(}\SpecialStringTok{f"type(course) = }\SpecialCharTok{\{}\BuiltInTok{type}\NormalTok{(course)}\SpecialCharTok{\}}\SpecialStringTok{"}\NormalTok{)  }\CommentTok{\# Expected: \textless{}class \textquotesingle{}str\textquotesingle{}\textgreater{}}
\BuiltInTok{print}\NormalTok{(}\SpecialStringTok{f"type(num\_sections) = }\SpecialCharTok{\{}\BuiltInTok{type}\NormalTok{(num\_sections)}\SpecialCharTok{\}}\SpecialStringTok{"}\NormalTok{)  }\CommentTok{\# Expected: \textless{}class \textquotesingle{}int\textquotesingle{}\textgreater{}}
\BuiltInTok{print}\NormalTok{(}\SpecialStringTok{f"num\_sections * section\_capacity = }\SpecialCharTok{\{}\NormalTok{num\_sections }\OperatorTok{*}\NormalTok{ section\_capacity}\SpecialCharTok{\}}\SpecialStringTok{"}\NormalTok{)  }\CommentTok{\# Expected: 125}

\CommentTok{\# Step 5: Calculate total course capacity}
\NormalTok{course\_capacity }\OperatorTok{=}\NormalTok{ num\_sections }\OperatorTok{*}\NormalTok{ section\_capacity  }\CommentTok{\# 5 × 25 = 125}
\BuiltInTok{print}\NormalTok{(}\SpecialStringTok{f"}\CharTok{\textbackslash{}n}\SpecialStringTok{Final Results:"}\NormalTok{)}
\BuiltInTok{print}\NormalTok{(}\SpecialStringTok{f"Course: }\SpecialCharTok{\{}\NormalTok{course}\SpecialCharTok{\}}\SpecialStringTok{"}\NormalTok{)}
\BuiltInTok{print}\NormalTok{(}\SpecialStringTok{f"Total sections: }\SpecialCharTok{\{}\NormalTok{num\_sections}\SpecialCharTok{\}}\SpecialStringTok{"}\NormalTok{)}
\BuiltInTok{print}\NormalTok{(}\SpecialStringTok{f"Capacity per section: }\SpecialCharTok{\{}\NormalTok{section\_capacity}\SpecialCharTok{\}}\SpecialStringTok{"}\NormalTok{)}
\BuiltInTok{print}\NormalTok{(}\SpecialStringTok{f"Total course capacity: }\SpecialCharTok{\{}\NormalTok{course\_capacity}\SpecialCharTok{\}}\SpecialStringTok{ students"}\NormalTok{)}
\end{Highlighting}
\end{Shaded}

\begin{verbatim}
Testing type() function:
type(course) = <class 'str'>
type(num_sections) = <class 'int'>
num_sections * section_capacity = 125

Final Results:
Course: PSTAT 5A
Total sections: 5
Capacity per section: 25
Total course capacity: 125 students
\end{verbatim}

\begin{center}\rule{0.5\linewidth}{0.5pt}\end{center}

\subsection{Summary of Key Concepts
Learned}\label{summary-of-key-concepts-learned}

\textbf{✅ JupyterHub Environment}

\begin{itemize}
\item
  Creating and renaming notebooks
\item
  Understanding cell types (Code vs Markdown)
\item
  Running cells with ▶️ button or \texttt{Shift+Enter}
\item
  Navigating the interface
\end{itemize}

\textbf{✅ Python Basics}

\begin{itemize}
\item
  Arithmetic operations: \texttt{+}, \texttt{-}, \texttt{*}, \texttt{/},
  \texttt{**}
\item
  Order of operations: Parentheses, Exponents, Multiplication/Division,
  Addition/Subtraction
\item
  Error reading: Understanding SyntaxError and NameError messages
\end{itemize}

\textbf{✅ Variables and Data Types}

\begin{itemize}
\item
  Variable assignment: \texttt{variable\_name\ =\ value}
\item
  Case sensitivity: \texttt{my\_var\ ≠\ My\_var}
\item
  Basic types: \texttt{int}, \texttt{float}, \texttt{str}, \texttt{bool}
\item
  Type checking: \texttt{type()} function
\end{itemize}

\textbf{✅ Modules and Imports}

\begin{itemize}
\item
  Import syntax: \texttt{from\ module\ import\ *} or
  \texttt{import\ module}
\item
  Using functions: After importing, functions become available
\item
  Math module: Contains mathematical functions like \texttt{sin()},
  \texttt{cos()}, etc.
\end{itemize}

\textbf{✅ Comments and Documentation}

\begin{itemize}
\item
  Inline comments: \texttt{\#\ This\ is\ a\ comment}
\item
  Block comments: \texttt{"""Multi-line\ comment"""}
\item
  Purpose: Document code for yourself and others
\end{itemize}

\textbf{✅ Programming Best Practices}

\begin{itemize}
\item
  Write descriptive variable names
\item
  Add comments to explain complex logic
\item
  Test your code incrementally
\item
  Read and understand error messages
\item
  Use existing variables in calculations when possible
\end{itemize}

\begin{center}\rule{0.5\linewidth}{0.5pt}\end{center}

\subsection{Next Steps}\label{next-steps}

In Lab 2, you'll learn about:

\begin{itemize}
\item
  Python functions and how to create them
\item
  Data structures (lists, dictionaries)
\item
  Control flow (if statements, loops)
\item
  More advanced programming concepts
\end{itemize}

\textbf{Great work completing Lab 1!} You now have the foundation needed
for statistical programming in Python.




\end{document}
